\chapter{已知问题}

这份模板还有很多欠缺的地方,有些问题虽然被发现了,
但目前还没找到满意的解决方法。本章记录模板还没解决的问题,希望用户能避开这些坑。
如果您有解决这些问题的思路,欢迎在 Github 上提交 issue 或 pull request。

\section{本模板的问题}

\subsection{英文 in 中文字体}

本模板部分英文的字体,与 Word 模板里不一样。

虽然要求里说,英文的字体要用 Times New Roman\textsuperscript{\textregistered},
但 Word 模板里,不少地方的英文、数字的字体不是 Times New Roman\textsuperscript{\textregistered},
而是旁边中文的字体。
比如正文的各级标题,还有本科封面,在 Word 模板里,这些西文的字体设置成“使用中文字体”。

这个问题应该能完美解决,但不知具体怎么做。

\subsection{行间距}

行间距的问题由两方面造成:不知道 Word 是如何计算行间距;以及,不知道 \LaTeX 如何准确设置行间距。
这些不能准确计算的行间距目前只能通过测量来确定。

% 虽然理论上,《论文撰写要求》就应该完整地描述论文的格式。但实际上,……

Word 排版算法经过高人的研究\footnote{\url{https://www.zhihu.com/question/26397264/answer/48165229}},
基本能找到《论文撰写要求》没有规定格式部分的准确算法,
但本科论文的页眉,由于插入了一张图片,实在没看懂 Word 是怎么计算行高的。

\LaTeX 方面,貌似它会试图对齐所有页面的第一行?
在页首插入垂直空格时,会插入多余的空间。
还有,如果正文的各级标题出现在页首,那么它们的 \texttt{before} 段前间隔会被忽略掉。

\subsection{字号}

由于行间距的问题,请尽量避免使用规定字号以外的大小。

\LaTeX 貌似不能像 Word 那样设置成固定行间距,并且简单(随意)改全局字号、行间距可能
会引发各种奇怪问题,所以本模板只通过 ctex 的接口,设置了 \texttt{zihao} 和 \texttt{linespread},
实现近似模板中的行间距效果。

\subsection{段落左侧缩进}

本科生正文要求段落左侧缩进2~字符,但\LaTeX 貌似有点麻烦,网上提到的方法大致可以分为两种:
一种把正文全都都加到一个环境里,另一种是每段前加一个命令来缩进。这两种都不是理想的解决方法,
参见讨论\footnote{\url{https://github.com/JackWzh/nuaathesis/issues/7}}。

\section{其他常见问题}

\subsection{宋体的粗体}\label{txt:issue:boldsun}

宋体的粗体可能会被换成黑体。
比如“\verb|右边是\textbf{粗宋体}|”,排版出来可能是“{右边是\heiti{粗宋体}}”。

这个问题源自很多宋体不带粗体,ctex 预置的6个字库中,只有 Fandol 和方正字库带有宋体的粗体。
但关键是 ctex 将其他字库中宋体的粗体,映射成黑体了。

解决方法有很多,一个方法是将代码写成“\verb|右边是\textbf{\songti 粗宋体}|”,
如果开启了 \texttt{AutoFakeBold},就可以伪粗体的宋体了。

另一个方法是使用 Fandol 字库或方正字库,不过它们的粗宋体比 Word 论文模板里的要粗/重很多,
特别是作为标题使用时,差异会很大。

还有一些方法,比如修改字体映射之类,应该可行的样子,有兴趣的读者可以尝试研究一下。
