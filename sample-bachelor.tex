\documentclass[
  degree=bachelor,
  type=paper,
  oneside,openany,
  AutoFakeBold=2.5
]{nuaathesis}
% 可用选项:
% degree=[bachelor|master|doctor]   必选,毕业论文学位
% type=[paper|design]   论文类型,本科必选,硕/博默认(强制)论文
% jincheng    可选,将学校信息设置为金城学院
% blankleft   可选,在启用 openright 时,去除空白页的页眉页脚,变成真正的白纸
% 其余选项将传给 ctexbook,比较有用的选项有:
% oneside|twoside       可选,单面或双面打印
% openany|openright     可选,在双面打印时,每章的第一页将打印在右侧
% AutoFakeBold          可选,貌似这个数值最接近 M$ Word 的粗体效果
% 推荐:双面打印时,使用 twoside,openright,blankleft;其他场合 oneside,openany

% 写作时,使用这个命令只渲染你想查看的部分,提升工作效率,定稿时注释掉整行
% \includeonly{chapter/00-cover,chapter/30-issues,}

\begin{document}

% 首先,导入论文的基础信息
%# -*- coding: utf-8-unix -*-

% 中文文档信息
\nuaaset{
  title = {\nuaathesis 示例文档}, % 论文题目
  author = {\nuaathesis~Group},   % 作者
  studentid = {012345678},  % 学号(本科)
  college = {\TeX 学院},    % 学院,或者(金城)院部
  major = {\LaTeX},     % 专业(本科)
  classid = {0123456},  % 班号(本科)
  advisors = {Donald Knuth\quad 大师},  % 指导教师
  libraryclassid = {TP371},       % 中图分类号(硕博)
  subjectclassid = {080605},      % 学科分类号(硕博)
  thesisid = {1028704 17-S036},   % 论文编号(硕博)
  majorsubject = {学位论文排版},  % 学科、专业(硕博)
  researchfield = {排版引用},     % 研究方向(硕博)
  % applydate = {二〇一七年七月}  % 默认当前日期
}

% 英文文档信息(本科可以不用写)
\nuaasetEn{
  college = {College of \TeX},
  title = {\nuaathesis~Tutorial},
  majorsubject = {Thesis Typesetting},
  author = {\nuaathesis~Group},
  advisors = {Prof.~Donald Knuth},
  % applydate = {July, 2017}
}

%% 中文摘要
\begin{abstract}
本文介绍如何使用\nuaathesis 文档类撰写南京航空航天大学学位论文。
\end{abstract}
\keywords{\TeX, \LaTeX, \nuaathesis}

%% 英文摘要
\begin{abstractEn}
A brief example of \nuaathesis.
\end{abstractEn}
\keywordsEn{\TeX, \LaTeX, \nuaathesis}


\makecover    % 封面
\makedeclare  % 承诺书

\frontmatter  % 启用页眉页脚
\makeabstract % 摘要
\nuaatableofcontents  % 正文目录

\mainmatter   % 以下正文

%# -*- coding: utf-8-unix -*-
\chapter{简介}\label{chap:intro}

这是南京航空航天大学(非官方)本科生学位论文\LaTeX 模板,当前版本是\version。本模板由\nuaathesis~Group共同开发,模板文档由Old Jack撰写。

本模板最早可以追溯到人人网上的一篇博客\footnote{\url{http://bit.ly/2rGOhf6}},由黄大宁、邓欣珂、徐添豪、石坤四人共同开发完善,参考了当时东南大学的\seuthesix 模板;除此之外在Github上也可以找到一个repo\footnote{\url{https://github.com/nuaa803/nuaa-thesis}},由Felix Ding、Jun Wang、Jackie Hou三位老师和Vevi Zhong同学共同维护,但是repo中的.cls和.sty文件是空文件。

回顾人人网的模板,没有直接提供nuaa.png和nuaa.bst文件,可以使用强制编译的方法生成文件,但是缺少左上角南航字样,参考文献格式也不符合标准。除此之外,旧模板使用了已经被放弃使用的CJK宏包,因此在编译\verb+\Unicode{}+命令时会出错,代码的阅读性和维护性也不如现在的ctex和xeCJK。由于上述原因,许多初次使用\LaTeX 和使用经验不多的同学,在一开始就放弃了使用旧版模板进行毕业设计的书写及排版。

基于南航无可用\LaTeX 学位模板可用的现状,\nuaathesis~Group基于旧\oldnuaathesis 模板、现东南大学的\seuthesix 模板和上海交通大学的SJTU Thesis模板,对模板进行了二次开发,基本实现了学士学位论文的模板。\nuaathesis~Group现有成员短期内不计划开发团队报告、硕士博士学位论文等其他文档的\LaTeX 模板,留给后续成员以及其他有需要、有能力的南航学子以后开发。

现在\nuaathesis 模板的代码托管在Github\footnote{\url{https://github.com/jackwzh/nuaathesis}}上,如有修改建议或者其他要求欢迎在Github上开issue或提pull request,\nuaathesis~Group会尽快回复,并酌情处理您的要求。

本模板基于Windows~10平台开发,使用MiKTeX v2.9发行版,所使用的宏包均跟进到最新版本。Linux平台由张一白使用\TeX Live测试,macOS平台由王成欣进行了测试,目前尚未出现任何问题。本模板尚未在Windows平台使用\CTeX / \TeX Live进行测试,如出现问题,请自行Google、Bing、Baidu搜索解决方法。学会使用搜索引擎、熟练阅读外文是一个学生最基本的能力,更是一个\LaTeX 使用者得以立足和前进的根本。

\nuaathesis~Group非常欢迎有其他南航的\LaTeX 使用者加入到本模板的开发与维护当中来,不断完善模板,为南航广大学子造福!

\section{模板使用}
\subsection{准备工作}
\begin{itemize}
  \item \TeX 发行版:Windows 系统推荐使用MiKTeX和\TeX Live这两种发行版,前者占用空间小,只在有宏包缺失情况下才进行下载,后者占用空间大,但基本无需担心宏包缺失。Linux系统(Arch系除外)推荐手动安装\TeX Live发行版,官方源中的TeXLive版本跟进较慢。macOS可以参考SJTU Thesis中的介绍。
  \item \TeX 知识:本说明文档提供\TeX 使用的例子,但不能解决所有的问题,因此使用前请自行学习\TeX~\&~\LaTeX 相关知识。
\end{itemize}

\subsection{模板编译}

\textbf{切记使用\XeLaTeX 进行编译。}使用\XeLaTeX 和\hologo{BibTeX}在文档中加入参考文献的流程可参考如下命令:

\begin{lstlisting}[basicstyle=\small\ttfamily, caption=手动逐次编译, numbers=none]
xelatex -no-pdf .tex文件名
biber --debug .tex文件名
xelatex .tex文件名
xelatex .tex文件名
\end{lstlisting}

使用\XeLaTeX 引擎编译可以直接通过各\LaTeX 编辑器实现,如:TeXworks,TeXmaker,TeXStudio,Emacs+插件,Atom+插件等等。biber命令需使用Windows的cmd/Power Shell、Linux和macOS下的bash实现。Windows平台可以自行编写简单的.bat批处理文件来实现。

使用biber需注意:\textbf{.bib文件内的文件记录必须在.tex文件中被引用,不引用的记录不要因为懒而不去除,否则将编译失败}

目录内容需要编译两次才能正常显示,原因推断为早期的电脑内存不够,所以将目录的生成分成了两步来进行。

\subsection{模板文件结构}
\begin{itemize}[noitemsep,topsep=0pt,parsep=0pt,partopsep=0pt]
  \item .tex文件:主文件,chapter下有各个章节的文件,强烈建议将文章模块化,方便调试与版本管理。
  \item .cls、.cfg文件:模板定义文件
  \item .bib文件:参考文献数据库文件
  \item figure文件夹:存放要插入的图片,其中nuaa.png不可删除
\end{itemize}

%# -*- coding: utf-8-unix -*-

\chapter{\LaTeX 排版用例}\label{chap:example}

本部分可参考SJTU Thesis模板\cite{SJTUThesis}的用例,源代码存放在tex文件夹下的examples.tex文件中,非常详尽。我们对其中几个部分有一些自己的见解,另外文献中也有一些没有提及的内容,写在本文档中。

\section{流程图}

对于不想花时间学习tizk宏包的同学,我们推荐在Power Point中绘制好流程图,然后导出pdf格式插入到文档中,学习成本较低,效果也非常好。

\section{表格}

SJTU Thesis中的表格介绍非常详尽,但其实有非常简单的从Excel生成\LaTeX 表格代码的方式,即Excel宏——excel2latex\footnote{\url{https://www.ctan.org/pkg/excel2latex?lang=en}}。此宏非常强大,可配合xcolor包生成有底色的表格。强烈推荐大家使用,提升撰写效率。

\begin{savenotes}
%% LaTeX通常不推荐在表格中插入脚注。但是,通过使用savenotes + makesavenoteenv 组合可在表格中插入脚注。请参考:https://en.wikibooks.org/wiki/LaTeX/Footnotes_and_Margin_Notes#Common_problems_and_workarounds

表\ref{tab:bicap}是张一白提供的一个\textbf{双语标题}和\textbf{在标题中使用脚注}的示例,对标题有特殊需求的同学可以参考此处的源代码:
\begin{table}[H]
  \centering
  \bicaption[chinese]{一个颇为标准的三线表格\footnote{这个例子来自\href{http://www.ctan.org/tex-archive/macros/latex/contrib/booktabs/booktabs.pdf}{《Publication quality tables in LATEX》}(booktabs宏包的文档)。这也是一个在表格中使用脚注的例子,请留意与threeparttable实现的效果有何不同。}}[english]{A standard three-line table\footnote{该表格演示了如何使用bicaption插入双语标题}}
  \label{tab:bicap}
  \begin{tabular}{@{}llr@{}} \toprule
    \multicolumn{2}{c}{Item} \\ \cmidrule(r){1-2}
    Animal & Description & Price (\$)\\ \midrule
    Gnat & per gram & 13.65 \\
    & each & 0.01 \\
    Gnu & stuffed & 92.50 \\
    Emu & stuffed & 33.33 \\
    Armadillo & frozen & 8.99 \\ \bottomrule
  \end{tabular}
\end{table}

\end{savenotes}

\LaTeX 中表格的使用体验比Word差很多,很遗憾这是不可避免的。除了使用上文提到的excel2latex宏之外,在线表格转换工具\footnote{http://www.tablesgenerator.com}也不失为一种高效的excel表格至\LaTeX 的转换方案。此外,也可以在word中打好表格,然后截图插入论文也不失为一种方案。

对于较长较大的表格,可以参考\LaTeX 笔记——lnotes2\footnote{\url{}http://dralpha.altervista.org/zh/tech/lnotes2.pdf}中的longtable(跨页表格)和sidewaystable(横向表格)等表格环境进行实现。另外可以使用\verb!p{2pt}!替代表格中的rcl,来控制表格每一列的宽度。

\section{图片表格指定位置插入}

图片和表格的插入默认是htbp四个选项,有时候这会让图片表格遍布整篇论文,可能会有同学非常反感这种情况,为了强制在当前位置插入图片,可以使用float宏包,然后使用H选项:\verb+\begin{figure}[H]+即可强制\TeX 在当前位置插入图片,从而避免正文和图片表格相距太远。

\section{多列图片}
由王成欣提供方案及示例代码。如需对两幅或多幅图片进行横向排版,建议使用subcaption包里的subfigure功能。效果如下:

\begin{figure}[H]
\centering
\begin{subfigure}{.45\textwidth}
  \centering
  \includegraphics[width=.9\linewidth]{nuaa-logo.pdf}
  \caption{左图}
  \label{fig:test_subfigure1}
\end{subfigure}
\begin{subfigure}{.45\textwidth}
  \centering
  \includegraphics[width=.9\linewidth]{nuaa-logo.pdf}
  \caption{右图}
  \label{fig:test_subfigure2}
\end{subfigure}
\caption{这是一个并列子图}
\label{fig:test_subfigure}
\end{figure}

请注意每行subfigure宽度的总和尽量不要超过一个\textbackslash textwidth,否则图像会自动折叠至下一行。

\chapter{一些来自 CQUThesis 的使用样例}

\section{字体命令}\label{txt:FreqCmd}
{\kaishu 玲珑骰子安红豆,入骨相思知不知。\hfill ——温庭筠}

{\fangsong 愿得一心人,白头不相离。\hfill ——卓文君}

{\ifcsname youyuan\endcsname\youyuan\else[无 \texttt{youyuan} 字体。]\fi 去年今日此门中,人面桃花相映红。\hfill ——崔护}

{\heiti 入我相思门,知我相思苦。\hfill ——李白}

{\ifcsname lishu\endcsname\lishu\else[无 \texttt{lishu} 字体。]\fi 此情可待成追忆?只是当时已惘然。\hfill ——李商隐}

{\songti 雨打梨花深闭门,忘了青春,误了青春。\hfill ——唐寅}

使用\texttt{textbf}和\texttt{textit}以及\texttt{underline}的效果分别如下:

这句话的\textbf{文字}分别\textit{使用}了三种命令来\underline{处理}。

The \textbf{words} in this sentences are \textit{processed} with three different \underline{cmd}.

(注:粗宋体可能会被替换成黑体,参见 \autoref{txt:issue:boldsun})


\section{表格样本}

\subsection{基本表格}
\label{sec:basictable}

我们经常会在表格下方标注数据来源,或者对表格里面的条目进行解释。前面的脚注是一种不错的方法,如果不喜欢脚注,可以在表格后面写注释,比如\autoref{tab:tabexamp1}。
\begin{table}[htbp]
    \centering
    \bicaption{复杂表格示例}{A more structured table}
    \label{tab:tabexamp1}
    \begin{minipage}[t]{0.8\textwidth}
    \begin{tabularx}{\linewidth}{|l|X|X|X|X|}
        \hline
        \multirow{2}*{\diagbox[width=5em]{x}{y}} & \multicolumn{2}{c|}{First Half} & \multicolumn{2}{c|}{Second Half}\\\cline{2-5}
        & 1st Qtr &2nd Qtr&3rd Qtr&4th Qtr \\ \hline
        East$^{*}$ &   20.4&   27.4&   90&     20.4 \\
        West$^{**}$ &   30.6 &   38.6 &   34.6 &  31.6 \\ \hline
    \end{tabularx}\\[2pt]
    \footnotesize
    *:东部\\
    **:西部
    \end{minipage}
\end{table}

此外,\autoref{tab:tabexamp1} 同时还演示了另外两个功能:1)通过 \texttt{tabularx} 的\texttt{|X|} 扩展实现表格自动放大;2)通过命令 \texttt{diagbox} 在表头部分插入反斜线。

\autoref{tab:performance} 是一个很长的表格。

\begin{longtable}[c]{c*{6}{r}}
    \bicaption[实验数据]{实验数据,这个题注是双语的,而且十分的长,注意这在索引中的处理方式}[Data in experiment]{Data in experiment, and this is a really long long long long long long long long long long long long long long text.}\label{tab:performance}\\
    \toprule
    测试程序 & \multicolumn{1}{c}{正常运行} & \multicolumn{1}{c}{同步} & \multicolumn{1}{c}{检查点} & \multicolumn{1}{c}{卷回恢复}
    & \multicolumn{1}{c}{进程迁移} & \multicolumn{1}{c}{检查点} \\
    & \multicolumn{1}{c}{时间 (s)}& \multicolumn{1}{c}{时间 (s)}&
    \multicolumn{1}{c}{时间 (s)}& \multicolumn{1}{c}{时间 (s)}& \multicolumn{1}{c}{
        时间 (s)}&  文件 (KB) \\\midrule
    \endfirsthead
    \multicolumn{7}{c}{续表~\thetable\hskip1em 实验数据}\\
    \toprule
    测试程序 & \multicolumn{1}{c}{正常运行} & \multicolumn{1}{c}{同步} & \multicolumn{1}{c}{检查点} & \multicolumn{1}{c}{卷回恢复}
    & \multicolumn{1}{c}{进程迁移} & \multicolumn{1}{c}{检查点} \\
    & \multicolumn{1}{c}{时间 (s)}& \multicolumn{1}{c}{时间 (s)}&
    \multicolumn{1}{c}{时间 (s)}& \multicolumn{1}{c}{时间 (s)}& \multicolumn{1}{c}{
        时间 (s)}&  文件 (KB) \\\midrule
    \endhead
    \hline
    \multicolumn{7}{r}{续下页}
    \endfoot
    \endlastfoot
    CG.A.2 & 23.05 & 0.002 & 0.116 & 0.035 & 0.589 & 32491 \\
    CG.A.4 & 15.06 & 0.003 & 0.067 & 0.021 & 0.351 & 18211 \\
    CG.A.8 & 13.38 & 0.004 & 0.072 & 0.023 & 0.210 & 9890 \\
    CG.B.2 & 867.45 & 0.002 & 0.864 & 0.232 & 3.256 & 228562 \\
    CG.B.4 & 501.61 & 0.003 & 0.438 & 0.136 & 2.075 & 123862 \\
    CG.B.8 & 384.65 & 0.004 & 0.457 & 0.108 & 1.235 & 63777 \\
    MG.A.2 & 112.27 & 0.002 & 0.846 & 0.237 & 3.930 & 236473 \\
    MG.A.4 & 59.84 & 0.003 & 0.442 & 0.128 & 2.070 & 123875 \\
    MG.A.8 & 31.38 & 0.003 & 0.476 & 0.114 & 1.041 & 60627 \\
    MG.B.2 & 526.28 & 0.002 & 0.821 & 0.238 & 4.176 & 236635 \\
    MG.B.4 & 280.11 & 0.003 & 0.432 & 0.130 & 1.706 & 123793 \\
    MG.B.8 & 148.29 & 0.003 & 0.442 & 0.116 & 0.893 & 60600 \\
    LU.A.2 & 2116.54 & 0.002 & 0.110 & 0.030 & 0.532 & 28754 \\
    LU.A.4 & 1102.50 & 0.002 & 0.069 & 0.017 & 0.255 & 14915 \\
    LU.A.8 & 574.47 & 0.003 & 0.067 & 0.016 & 0.192 & 8655 \\
    LU.B.2 & 9712.87 & 0.002 & 0.357 & 0.104 & 1.734 & 101975 \\
    LU.B.4 & 4757.80 & 0.003 & 0.190 & 0.056 & 0.808 & 53522 \\
    LU.B.8 & 2444.05 & 0.004 & 0.222 & 0.057 & 0.548 & 30134 \\
    CG.B.2 & 867.45 & 0.002 & 0.864 & 0.232 & 3.256 & 228562 \\
    CG.B.4 & 501.61 & 0.003 & 0.438 & 0.136 & 2.075 & 123862 \\
    CG.B.8 & 384.65 & 0.004 & 0.457 & 0.108 & 1.235 & 63777 \\
    MG.A.2 & 112.27 & 0.002 & 0.846 & 0.237 & 3.930 & 236473 \\
    MG.A.4 & 59.84 & 0.003 & 0.442 & 0.128 & 2.070 & 123875 \\
    MG.A.8 & 31.38 & 0.003 & 0.476 & 0.114 & 1.041 & 60627 \\
    MG.B.2 & 526.28 & 0.002 & 0.821 & 0.238 & 4.176 & 236635 \\
    MG.B.4 & 280.11 & 0.003 & 0.432 & 0.130 & 1.706 & 123793 \\
    MG.B.8 & 148.29 & 0.003 & 0.442 & 0.116 & 0.893 & 60600 \\
    LU.A.2 & 2116.54 & 0.002 & 0.110 & 0.030 & 0.532 & 28754 \\
    LU.A.4 & 1102.50 & 0.002 & 0.069 & 0.017 & 0.255 & 14915 \\
    LU.A.8 & 574.47 & 0.003 & 0.067 & 0.016 & 0.192 & 8655 \\
    LU.B.2 & 9712.87 & 0.002 & 0.357 & 0.104 & 1.734 & 101975 \\
    LU.B.4 & 4757.80 & 0.003 & 0.190 & 0.056 & 0.808 & 53522 \\
    LU.B.8 & 2444.05 & 0.004 & 0.222 & 0.057 & 0.548 & 30134 \\
    EP.A.2 & 123.81 & 0.002 & 0.010 & 0.003 & 0.074 & 1834 \\
    EP.A.4 & 61.92 & 0.003 & 0.011 & 0.004 & 0.073 & 1743 \\
    EP.A.8 & 31.06 & 0.004 & 0.017 & 0.005 & 0.073 & 1661 \\
    EP.B.2 & 495.49 & 0.001 & 0.009 & 0.003 & 0.196 & 2011 \\
    EP.B.4 & 247.69 & 0.002 & 0.012 & 0.004 & 0.122 & 1663 \\
    EP.B.8 & 126.74 & 0.003 & 0.017 & 0.005 & 0.083 & 1656 \\
    \bottomrule
\end{longtable}

\section{定理环境}
\label{sec:theorem}

给大家演示一下各种和证明有关的环境:

\begin{assumption}
    假设以下数学方程成立:
    \begin{eqnarray}
    \label{eq:eqnxmp}
    c & = & a^2 - b^2\\
    & = & (a+b)(a-b)
    \end{eqnarray}
\end{assumption}

\begin{assumption}
    依然假设以下数学方程成立,注意整个系统是自动编号的:
    \begin{eqnarray}
    \label{eq:eqnxmp2}
    c & = & a^2 - b^2\\
    & = & (a+b)(a-b)
    \end{eqnarray}
\end{assumption}

\begin{cor}
    四川话配音的《猫和老鼠》是世界上最好看最好听最有趣的动画片。
    \begin{alignat}{3}
    V_i & =v_i - q_i v_j, & \qquad X_i & = x_i - q_i x_j,
    & \qquad U_i & = u_i,
    \qquad \text{for $i\ne j$;}\label{eq:B}\\
    V_j & = v_j, & \qquad X_j & = x_j,
    & \qquad U_j & u_j + \sum_{i\ne j} q_i u_i.
    \end{alignat}
\end{cor}

迢迢牵牛星,皎皎河汉女。
纤纤擢素手,札札弄机杼。
终日不成章,泣涕零如雨。
河汉清且浅,相去复几许。
盈盈一水间,脉脉不得语。

\begin{exmp}
    大家来看\autoref{ktc}。
    \begin{equation}
    \label{ktc}
    \left\{\begin{array}{l}
    \nabla f({\mbox{\boldmath $x$}}^*)-\sum\limits_{j=1}^p\lambda_j\nabla g_j({\mbox{\boldmath $x$}}^*)=0\\[0.3cm]
    \lambda_jg_j({\mbox{\boldmath $x$}}^*)=0,\quad j=1,2,\cdots,p\\[0.2cm]
    \lambda_j\ge 0,\quad j=1,2,\cdots,p.
    \end{array}\right.
    \end{equation}
\end{exmp}


\section{参考文献}
\label{sec:bib}
重庆大学的要求是参考文献以上标的形式标注于论述之后,就像这样:

研究表明\cite{r1},早睡早起有益身体健康。

如果想同时引用多个文献\cite{r2,r3,r4,r6},只需要在\verb|cite{}|中用逗号分开\textsf{citeKey}就好。
CQUThesis 同时提供正文模式的参考文献引用功能\texttt{inlinecite},适用于以下情况:
文献\inlinecite{r6}表明,文献\inlinecite{r7,r8,r9}所述的情况是有理论依据的。

\chapter{已知问题}

这份模板还有很多欠缺的地方,有些问题虽然被发现了,
但目前还没找到满意的解决方法。本章记录模板还没解决的问题,希望用户能避开这些坑。
如果您有解决这些问题的思路,欢迎在 Github 上提交 issue 或 pull request。

\section{本模板的问题}

\subsection{英文 in 中文字体}

本模板部分英文的字体,与 Word 模板里不一样。

虽然要求里说,英文的字体要用 Times New Roman\textsuperscript{\textregistered},
但 Word 模板里,不少地方的英文、数字的字体不是 Times New Roman\textsuperscript{\textregistered},
而是旁边中文的字体。
比如正文的各级标题,还有本科封面,在 Word 模板里,这些西文的字体设置成“使用中文字体”。

这个问题应该能完美解决,但不知具体怎么做。

\subsection{行间距}

行间距的问题由两方面造成:不知道 Word 是如何计算行间距;以及,不知道 \LaTeX 如何准确设置行间距。
这些不能准确计算的行间距目前只能通过测量来确定。

% 虽然理论上,《论文撰写要求》就应该完整地描述论文的格式。但实际上,……

Word 排版算法经过高人的研究\footnote{\url{https://www.zhihu.com/question/26397264/answer/48165229}},
基本能找到《论文撰写要求》没有规定格式部分的准确算法,
但本科论文的页眉,由于插入了一张图片,实在没看懂 Word 是怎么计算行高的。

\LaTeX 方面,貌似它会试图对齐所有页面的第一行?
在页首插入垂直空格时,会插入多余的空间。
还有,如果正文的各级标题出现在页首,那么它们的 \texttt{before} 段前间隔会被忽略掉。

\subsection{字号}

由于行间距的问题,请尽量避免使用规定字号以外的大小。

\LaTeX 貌似不能像 Word 那样设置成固定行间距,并且简单(随意)改全局字号、行间距可能
会引发各种奇怪问题,所以本模板只通过 ctex 的接口,设置了 \texttt{zihao} 和 \texttt{linespread},
实现近似模板中的行间距效果。

\subsection{段落左侧缩进}

本科生正文要求段落左侧缩进2~字符,但\LaTeX 貌似有点麻烦,网上提到的方法大致可以分为两种:
一种把正文全都都加到一个环境里,另一种是每段前加一个命令来缩进。这两种都不是理想的解决方法,
参见讨论\footnote{\url{https://github.com/JackWzh/nuaathesis/issues/7}}。

\section{其他常见问题}

\subsection{宋体的粗体}\label{txt:issue:boldsun}

宋体的粗体可能会被换成黑体。
比如“\verb|右边是\textbf{粗宋体}|”,排版出来可能是“{右边是\heiti{粗宋体}}”。

这个问题源自很多宋体不带粗体,ctex 预置的6个字库中,只有 Fandol 和方正字库带有宋体的粗体。
但关键是 ctex 将其他字库中宋体的粗体,映射成黑体了。

解决方法有很多,一个方法是将代码写成“\verb|右边是\textbf{\songti 粗宋体}|”,
如果开启了 \texttt{AutoFakeBold},就可以伪粗体的宋体了。

另一个方法是使用 Fandol 字库或方正字库,不过它们的粗宋体比 Word 论文模板里的要粗/重很多,
特别是作为标题使用时,差异会很大。

还有一些方法,比如修改字体映射之类,应该可行的样子,有兴趣的读者可以尝试研究一下。

\include{chapter/40-duplicate}
%# -*- coding: utf-8-unix -*-
\chapter{模板更新记录}
\label{chap:updatelog}

\textbf{2017年6月22日} \nuaathesis 正式通过毕业设计审核,v1.0发布,增加毕业设计/毕业论文选项,并调整页眉;针对双面打印选项调整页脚;细节调整。

\textbf{2017年6月5日} v0.92发布,增加 biblatex 对 natbib 支持,如\verb!\citep!可以直接在行中引用编号, \verb!\citet!可以引用作者 (这里貌似仍然是个 bug, 理论上应该是引用题目,还没仔细研究。); 添加subcaption和caption包,修复bicaption参数; 添加多列图片示例代码;多处细节调整。

\textbf{2017年5月12日} v0.91a发布,添加双语标题和标题中使用脚注用例;增加几个默认宏包,方便使用;部分细节修调整。

\textbf{2017年3月15日} v0.91发布,使用开源Fandol字体替代华文字体和思源雅黑字体。

\textbf{2017年3月14日} v0.9a发布,加入使脚注出现在页脚线下方的代码,加入模板更新记录。

\textbf{2017年3月14日} v0.9跨版发布,代码重构,模板基本实现,开始由Git进行版本控制,进入微调阶段。

\textbf{2013年6月4日} v0.3发布,加入对团队报告的支持,加入几个宏包,加一些预定义符号。

\textbf{2013年5月29日} v0.2发布,详情未知。

\textbf{2013年5月18日} v0.1发布,详情未知。

\textbf{2013年5月15日} 模板发布,版本号v0.0。


\backmatter   % 正文后无编号部分,
\bibliographystyle{gbt-7714-2015-numerical}   % 参考文献的样式
\bibliography{bib/sample}   % 参考文献

%# -*- coding: utf-8-unix -*-
\chapter[致谢]{致\hskip\ccwd{}谢}
首先,谨代所有以后使用本模板排版毕设的南航学子感谢本模板v2.0的代码贡献者——yzwduck,v2.0的工作基本均由他一人完成,让\nuaathesis 有了质的飞跃,向他表示最高的敬意!

其次要感谢使用、测试本模板v1.0版之前所有版本代码,并反馈问题、提pull request的几位co-authors,
是他们让\nuaathesis v1.0 得以如此美丽。
他们是:1613的张一白\footnote{\url{https://github.com/summershrimp}}和曾宪文\footnote{\url{https://github.com/RexSkz}}、0313的王成欣\footnote{\url{https://github.com/cvcore}}、0513的Gavin Lee\footnote{\url{https://github.com/gavinlee1994}}。
我从他们那学习到了很多新的\LaTeX 知识,也发现在南航有着和我一样喜欢\LaTeX 的朋友,而且他们都是比我更优秀的同学,我非常高兴能在大学的最后阶段认识他们。

最后特别感谢\oldnuaathesis 的四位作者:黄大宁、邓欣珂、徐添豪、石坤,\seuthesix 的开发维护团队,以及SJTU Thesis的开发维护团队,我们向他们的模板借鉴了很多的源码,没有这些前辈的工作,就没有今天的\nuaathesis,代表所有使用\nuaathesis 的南航学子感谢你们!

% \appendix     % 附录部分
\chapter{附\hskip\ccwd{}录}

\section{v0.9a后记——Old Jack 的吐槽}

\verb!\begin{轻松+愉快}!

Old Jack 他有点累......

Old Jack 两年前就开始关注南航毕设的\LaTeX 模板了,但是两年了还没有任何有实际意义的新动作,所以Old Jack 就亲自操刀制作了新的一版。虽然很多代码都是从其他模板中直接摘抄过来的,但是这也是\TeX 最普遍、最快捷的学习\&开发方法。一开始 Old Jack 也想造轮子,但是轮子真的不好造。

在制作过程中遇到了几个关键性的问题:
\begin{itemize}
  \item 前文提到的三种粗体
  \item nuaa.png源文件和页眉制作
  \item 英文字母、章节标题莫名其妙的加粗
  \item 脚注相对页脚线的位置
\end{itemize}

第一个问题 Old Jack 曾经用\TeX 中伪粗体(FakeBold)的方法实现过,但是效果并不好,而且当时受到最后一个问题的强烈影响,不得不使用其他字体来解决这个问题。

第二个问题 Old Jack 开始是使用官方模板中的图片,但是分辨率太低,效果很差。于是 Old Jack Google以图搜图找到了现在的这个文件的源文件,经过了一系列不可描述的操作后得到了现在的 nuaa.png 。页眉的制作也让 Old Jack 很头疼,论文要求论文到顶端和底端的距离分别为2.5cm和2.0cm,Old Jack 很naive的就给geometry设置了这个数值,但是效果和官方模板差了很多,于是 Old Jack 只好一点一点地调试,达到了近似官方模板的效果。页脚和官方模板有细微的区别,Old Jack 认为这无伤大雅,是要罗马数字和阿拉伯数字编号正确应该就可以了。

第三个问题是一个非常奇怪的问题。使用伪粗体时所有标题全都加粗了,非常难看,经过了代码重构和不停地调试解决了这个问题。在模板完成99\% 后发现最后致谢中的英文字体全都加粗了, Old Jack 几次审视代码和调试都没有解决。偶然间,Old Jack 将全部主要文件全部提取出来,放入另一个文件夹,然后重新编译就解决了这个问题!当然后来发现代码中确实有一个地方有小问题\textbf{可能}会影响,但是这不是上一次出错的原因。Old Jack 对于各位使用模板的南航学子以及其他可能会参考此模板的\TeX 爱好者提了一个建议:\textbf{任何语言,任何代码出现莫名其妙的问题时,换一个文件夹,改一下名字,重新跑一下,可能会得到意想不到的结果。}当然这不是万能的解决方法。

第四个问题就如第一章中脚注和页脚线的情况,感觉两条线很别扭。 Old Jack 犹豫了很久,最后没有采用将脚注放在页脚线下的方案,因为 Old Jack 觉得还是两条线的方案好看。对于想要将脚注放在页脚线下方的同学,可以在主文件中取消注释那段代码,来实现所需要的效果。

Old Jack 他完成了模板的再制作,但是他没有心气再写出一篇能够指导大家使用\LaTeX 的文档了(好吧,Old Jack 他承认懒是一部分因素),望大家谅解 Old Jack。

\verb!\end{轻松+愉快}!

\section{v1.0后记}

Old Jack 非常高兴,因为他不是一个人在战斗。再次感谢张一白、王成欣、曾宪文、Gavin Lee等人的工作,没有他们,\nuaathesis 不会像现在这么美丽。

经过\nuaathesis~Group的努力和测试,\nuaathesis 迎来了v1.0版,也就是第一个正式发行版。一路走来也是有些坎坷,各种各样的小问题一直困扰着我们,其实v1.0 也还有着一些细小的问题尚未解决。不过Old Jack请大家放心,这些小问题不影响模板的使用。很多已经被我们解决的小问题比如页眉的大小位置,中英文字体是否正确,摘要的章节标题不能是加粗的宋体等等,老师可能不去管这些,甚至注意不到有什么区别。相比之下,重要的地方是:公式、图表的编号,图表和文本的位置,参考文献的格式等等才是老师关注的点。很多地方只是我们几个人为了追求和office模板尽可能接近,才不断地进行修改调整,也是有点讽刺。

写毕设论文的时候,Old Jack 不止一次看到隔壁室友调公式内容,Mathtype和Office装了卸,卸了装、调公式编号、调标题位置和大小、调首行缩进、调段间距等等等等,看着他们搞得焦头烂额的,Old Jack 都觉得心累。打印时也是这样,有太多的人在打印店不停地修改自己的论文,有因为office和wps不兼容修改的,有office版本不兼容修改的,有因为页眉页脚错误修改的等等。然而 Old Jack 他在写论文时从来没有担心过这些事情(当然,作为模板开发者 Old Jack 确实操心了很多,2333),他也第一次真正体会到了什么叫做专注于内容,真的挺轻松的(表格是例外)。

对于模板的推广,Old Jack觉得使用人数仍然不会太多,毕竟\LaTeX 的群众基础太小,除了8院,其他学院对公式的需求整体来讲并不迫切,Old Jack 猜测大部分知道、了解\LaTeX 的同学是通过数学建模竞赛这个途径才学习了\LaTeX ;同时因为涉及到学习新的程序语言,时间成本也较大,所以很多同学的学习意愿不高。不过\nuaathesis 的目标人群本来也不是全校所有学生,Old Jack 的思路,Old Jack 相信也是\nuaathesis~Group其他开发者的思路是:
\begin{enumerate}
  \item 为自己服务,这是\nuaathesis~Group开发模板的第一动力;
  \item 对已经掌握\LaTeX 基本语法的同学,\nuaathesis~Group为他们在毕业设计时能更轻松地撰写论文,提供平台和机会;
  \item 对准备学习\LaTeX 以及已经学习了一点\LaTeX 的同学,\nuaathesis~Group为他们提供学下去的动力和平台。
\end{enumerate}

即将毕业了,回首大学四年, Old Jack 做过疯狂的事情,也找到了一份看起来还可以的工作,只觉得还没对学校做过什么有用的事情,尽管 Old Jack 对学校其实并不是很有感情。完成了这个模板后,至少 Old Jack 可以减少一个遗憾,然后离开学校了。虽然这不是什么惊天动地的工作,但是至少 Old Jack 做了件他认为还算有意义的事情。Old Jack应该还会再维护\nuaathesis 一段时间,期待有后继者能够接过火炬,继续完善并推广\nuaathesis 。

想说的可能也就这么多了,Old Jack out!

\hfill 0813~王志浩,2017.6.24

\section{未知版本的后记}

也是两年前开始关注南航毕设的\LaTeX 模板了,但直到毕业前,都没能去静下心来学习\LaTeX。

现在差不多本科毕业一年,或者说,一年后要开写硕士学位论文了,决定这次一定要用\LaTeX。
本打算照着 CQUThesis 来造轮子的时候,逛纸飞机看到 \nuaathesis~v1.0 发布了。
非常激动、也很自愧,同样是经历了大学四年的人,我没能把这模板做出来。

于是马上把两年前为了模板而画的校名(矢量图)传了上去\footnote{\url{https://github.com/nuaatug/nuaathesis/commit/24fa82e4f136f3a713859335a395ecf364a342a2}}。

原本打算在 v1.0 版的基础上修改的,但因为行间距设置有问题,封面与 Word 模板也有点差异,
还要再加入硕/博士的模板,于是干脆改成 \texttt{Documented LaTeX Source (.dtx)},
方便以后写模板的文档。

做模板过程中遇到的大问题,在于如何正确理解学校对论文格式的要求。
虽然有《本科毕业设计(论文)撰写格式要求》、《研究生学位论文撰写要求》,
但这些要求依然不够细致,因为那些要求都是假定你用 Word 来写论文的,要求里的内容是 Word 设置的操作方法,
所以还要先学习 Word 的排版算法。虽然这不是热门的资料,而且还有 CJK 独有的坑,
但幸好网上有高人已经把这问题基本解决了。

最后吐槽一下学校的 Word 模板,我觉得可以说,那个 Word 模板可能从最初做出来后,就基本没有变化。
那个“最初”或许可以追溯到上个世纪。很多编号的事情都要由手工来完成,比如说目录页码、
各级标题的编号、题注等。这些完全可以自动编号的工作,如果要手工做的话,【掀桌的emoji】。


\end{document}
