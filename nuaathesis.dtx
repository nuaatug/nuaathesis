% \iffalse meta-comment
%
% Copyright: TODO
% -------------------------------------------------------
%
% This file may be distributed and/or modified under the ...
%
% \fi
%
% \iffalse
%<*driver>
\ProvidesFile{nuaathesis.dtx}
\documentclass{ltxdoc}
\usepackage{dtx-style}

\EnableCrossrefs
\CodelineIndex
\RecordChanges

\begin{document}
  \DocInput{\jobname.dtx}
\end{document}
%</driver>
% \fi
%
% \CheckSum{0}
%
% \CharacterTable
%  {Upper-case    \A\B\C\D\E\F\G\H\I\J\K\L\M\N\O\P\Q\R\S\T\U\V\W\X\Y\Z
%   Lower-case    \a\b\c\d\e\f\g\h\i\j\k\l\m\n\o\p\q\r\s\t\u\v\w\x\y\z
%   Digits        \0\1\2\3\4\5\6\7\8\9
%   Exclamation   \!     Double quote  \"     Hash (number) \#
%   Dollar        \$     Percent       \%     Ampersand     \&
%   Acute accent  \'     Left paren    \(     Right paren   \)
%   Asterisk      \*     Plus          \+     Comma         \,
%   Minus         \-     Point         \.     Solidus       \/
%   Colon         \:     Semicolon     \;     Less than     \<
%   Equals        \=     Greater than  \>     Question mark \?
%   Commercial at \@     Left bracket  \[     Backslash     \\
%   Right bracket \]     Circumflex    \^     Underscore    \_
%   Grave accent  \`     Left brace    \{     Vertical bar  \|
%   Right brace   \}     Tilde         \~}
%
% \DoNotIndex{\newenvironment,\@bsphack,\@empty,\@esphack,\sfcode}
% \DoNotIndex{\addtocounter,\label,\let,\linewidth,\newcounter}
% \DoNotIndex{\noindent,\normalfont,\par,\parskip,\phantomsection}
% \DoNotIndex{\providecommand,\ProvidesPackage,\refstepcounter}
% \DoNotIndex{\RequirePackage,\setcounter,\setlength,\string,\strut}
% \DoNotIndex{\textbackslash,\texttt,\ttfamily,\usepackage}
% \DoNotIndex{\begin,\end,\begingroup,\endgroup,\par,\\}
% \DoNotIndex{\if,\ifx,\ifdim,\ifnum,\ifcase,\else,\or,\fi}
% \DoNotIndex{\let,\def,\xdef,\edef,\newcommand,\renewcommand}
% \DoNotIndex{\expandafter,\csname,\endcsname,\relax,\protect}
% \DoNotIndex{\Huge,\huge,\LARGE,\Large,\large,\normalsize}
% \DoNotIndex{\small,\footnotesize,\scriptsize,\tiny}
% \DoNotIndex{\normalfont,\bfseries,\slshape,\sffamily,\interlinepenalty}
% \DoNotIndex{\textbf,\textit,\textsf,\textsc}
% \DoNotIndex{\hfil,\par,\hskip,\vskip,\vspace,\quad}
% \DoNotIndex{\centering,\raggedright,\ref}
% \DoNotIndex{\c@secnumdepth,\@startsection,\@setfontsize}
% \DoNotIndex{\ ,\@plus,\@minus,\p@,\z@,\@m,\@M,\@ne,\m@ne}
% \DoNotIndex{\@@par,\DeclareOperation,\RequirePackage,\LoadClass}
% \DoNotIndex{\AtBeginDocument,\AtEndDocument}
%
%
% \GetFileInfo{\jobname.dtx}
%
% \changes{v1.0}{2017/6/22}{nuaathesis 正式通过毕业设计审核,v1.0发布,增加毕业设计/毕业论文选项,并调整页眉;针对双面打印选项调整页脚;细节调整。}
% \changes{v0.92}{2017/6/5}{v0.92发布,增加 biblatex 对 natbib 支持,如citep可以直接在行中引用编号, citet可以引用作者 (这里貌似仍然是个 bug, 理论上应该是引用题目,还没仔细研究。); 添加subcaption和caption包,修复bicaption参数; 添加多列图片示例代码;多处细节调整。}
% \changes{v0.91a}{2017/5/12}{v0.91a发布,添加双语标题和标题中使用脚注用例;增加几个默认宏包,方便使用;部分细节修调整。}
% \changes{v0.91}{2017/3/15}{v0.91发布,使用开源Fandol字体替代华文字体和思源雅黑字体。}
% \changes{v0.9a}{2017/3/14}{v0.9a发布,加入使脚注出现在页脚线下方的代码,加入模板更新记录。}
% \changes{v0.9}{2017/3/14}{v0.9跨版发布,代码重构,模板基本实现,开始由Git进行版本控制,进入微调阶段。}
% \changes{v0.3}{2013/6/4}{v0.3发布,加入对团队报告的支持,加入几个宏包,加一些预定义符号。}
% \changes{v0.2}{2013/5/29}{v0.2发布,详情未知。}
% \changes{v0.1}{2013/5/18}{v0.1发布,详情未知。}
% \changes{v0.0}{2013/5/15}{模板发布,版本号v0.0。}
%
% \def\indexname{代码索引}
% \def \glossaryname{Glossary Name Here}
% \IndexPrologue{\section{\indexname}}
% \GlossaryPrologue{\section{\glossaryname}}
%
% \title{The \textsf{nuaathesis} class\thanks{This document
%   corresponds to \textsf{nuaathesis}~\fileversion, dated \filedate.}}
% \author{Nanashi \\ \texttt{nanashi@nuaa.edu.cn}}
% \date{\fileversion\ (\filedate)}
%
% \maketitle
%
% \section{Introduction}
%
% Put text here.
%
% \section{Usage}
%
% Put text here.
%
% \DescribeMacro{\dummyMacro}
% This macro does nothing.\index{doing nothing|usage} It is merely an
% example.  If this were a real macro, you would put a paragraph here
% describing what the macro is supposed to do, what its mandatory and
% optional arguments are, and so forth.
%
% \DescribeEnv{dummyEnv}
% This environment does nothing.  It is merely an example.
% If this were a real environment, you would put a paragraph here
% describing what the environment is supposed to do, what its
% mandatory and optional arguments are, and so forth.
%
% \StopEventually{\PrintChanges\PrintIndex}
%
% \section{实现细节}
%
% \subsection{模板信息}
%    \begin{macrocode}
%<cls>\NeedsTeXFormat{LaTeX2e}
%<cls>\ProvidesClass{nuaathesis}
%<cfg>\ProvidesFile{nuaathesis.cfg}
%<cls|cfg>[2017/07/01 v0.02 NUAA Thesis Template]
%    \end{macrocode}
%
% 定义一些开发常量:
%    \begin{macrocode}
\def\version{0.02}
%    \end{macrocode}

% \subsection{配置文件}
%    \begin{macrocode}
%<*cfg>
%    \end{macrocode}
% 封面页上用到的常量:
%    \begin{macrocode}
\newcommand\nuaa@label@nuaa{南京航空航天大学}
\newcommand\nuaa@label@nuaajc{南京航空航天大学金城学院}
\newcommand\nuaa@label@worktype@paper{毕业论文}
\newcommand\nuaa@label@worktype@design{毕业设计}
\newcommand\nuaa@label@worktype@master{硕士学位论文}
\newcommand\nuaa@label@worktype@doctor{博士学位论文}
\newcommand\nuaa@label@thesisnum{编号}
\newcommand\nuaa@label@title{题\quad 目}
\newcommand\nuaa@label@teamname{团队名称}
\newcommand\nuaa@label@author{学生姓名}
\newcommand\nuaa@label@studentid{学 \hfill 号}
\newcommand\nuaa@label@college{学 \hfill 院}
\newcommand\nuaa@label@department{系 \hfill 部}
\newcommand\nuaa@label@major{专 \hfill 业}
\newcommand\nuaa@label@classid{班 \hfill 级}
\newcommand\nuaa@label@advisor{指 \hfill 导 \hfill 教 \hfill 师}
\newcommand\nuaa@label@researchername{研究生姓名}
\newcommand\nuaa@label@majorsubject{学科、专业}
\newcommand\nuaa@label@researchfield{研 \hfill 究 \hfill 方 \hfill 向}
\newcommand\nuaa@label@graduateschool{研究生院}
\newcommand\nuaa@labelEn@nuaa{Nanjing University of Aeronautics and Astronautics}
\newcommand\nuaa@labelEn@graduateschool{The Graduate School}
\newcommand\nuaa@labelEn@degreemaster{Master of Engineering}
\newcommand\nuaa@labelEn@degreedoctor{Doctor of Philosophy}
%    \end{macrocode}
% 摘要页用到的常量:
%    \begin{macrocode}
\newcommand\nuaa@label@abstract{摘\quad 要}
\newcommand\nuaa@label@abstractshort{摘要}
\newcommand\nuaa@label@keywords{关键词:}
\newcommand\nuaa@label@keywordsep{,}
\newcommand\nuaa@label@abstract@toc{摘要}
\newcommand\nuaa@labelEn@abstract{Abstract}
\newcommand\nuaa@labelEn@ABSTRACT{ABSTRACT}
\newcommand\nuaa@labelEn@KeyWords{Key Words: }
\newcommand\nuaa@labelEn@keywords{Keywords: }
\newcommand\nuaa@labelEn@keywordsep{; }
%    \end{macrocode}
% 其他地方用到的常量:
%    \begin{macrocode}
\newcommand\nuaa@label@reportpaper{报告纸}
\newcommand\nuaa@label@tableofcontents{目\quad 录}
\newcommand\nuaa@label@listoffigurestables{图表清单}
\newcommand\nuaa@label@figurename{图}
\newcommand\nuaa@label@tablename{表}
\newcommand\nuaa@label@lstlistingname{代码}
\newcommand\nuaa@labelEn@figurename{Fig.}
\newcommand\nuaa@labelEn@tablename{Table}

%</cfg>
%    \end{macrocode}

% \subsection{文档类的选项}
% \subsubsection{定义与处理}
% 使用 \pkg{kvoptions} 来处理传入 \texttt{nuaathesis} 的参数。
%    \begin{macrocode}
%<*cls>
\RequirePackage{kvoptions}
\SetupKeyvalOptions{
  family=nuaa,
  prefix=nuaa@,
  setkeys=\kvsetkeys
}
%    \end{macrocode}
% 定义用户类型,目前支持本科、硕士、博士三类。
%    \begin{macrocode}
\newif\ifnuaa@bachelor \nuaa@bachelorfalse
\newif\ifnuaa@master   \nuaa@masterfalse
\newif\ifnuaa@doctor   \nuaa@doctorfalse
\define@key{nuaa}{degree}{
  \expandafter\csname nuaa@#1true\endcsname}
%    \end{macrocode}
% 定义论文的主语言,目前支持中文,计划支持英文,尚未支持日语。
%    \begin{macrocode}
\newif\ifnuaa@lang@cn \nuaa@lang@cnfalse
\newif\ifnuaa@lang@en \nuaa@lang@enfalse
\newif\ifnuaa@lang@ja \nuaa@lang@jafalse
\define@key{nuaa}{lang}{
  \expandafter\csname nuaa@lang@#1true\endcsname}
%    \end{macrocode}
% 定义文档类型是毕业论文还是毕业设计。
%    \begin{macrocode}
\newif\ifnuaa@worktype@paper  \nuaa@worktype@paperfalse
\newif\ifnuaa@worktype@design \nuaa@worktype@designfalse
\define@key{nuaa}{type}{
  \expandafter\csname nuaa@worktype@#1true\endcsname}
%    \end{macrocode}
% 金城学院
%    \begin{macrocode}
\DeclareBoolOption[false]{jincheng}
%    \end{macrocode}
% 右开时空白的左页是否让页眉页脚空白
%    \begin{macrocode}
\DeclareBoolOption[false]{blankleft}
%    \end{macrocode}
% 双页模式下摘要页右开
%    \begin{macrocode}
\DeclareBoolOption[false]{abstractopenright}
%    \end{macrocode}
% 盲审模式开关
%    \begin{macrocode}
\DeclareBoolOption[false]{bindtrail}
%    \end{macrocode}
% 其他传递选项给 \texttt{ctexbook}
%    \begin{macrocode}
\DeclareDefaultOption{\PassOptionsToClass{\CurrentOption}{ctexbook}}
%    \end{macrocode}
% 处理选项
%    \begin{macrocode}
\kvsetkeys{nuaa}{}
\ProcessKeyvalOptions*
\AtEndOfClass{
%%  This is file `nuaathesix.cls',
%%  This file provides a LaTeX document class for typesetting thesis of Nanjing University of Aeronautics and Astronautics, Nanjing, China.
%%  Copyright (c) 2017 Jack Wang, email: jackwzh_wy@163.com

\ProvidesFile{nuaathesis.cfg}[2017/03/16 1.0 nuaathesis configuration file]

%%==============================
%% 目录、摘要、Abstract、符号列表
%%==============================
\newcommand{\nuaa@abstractname}{摘\quad 要}
\newcommand{\nuaa@abstractnameen}{Abstract}
\newcommand{\nuaa@terminologytitle}{符号列表}
\newcommand{\nuaa@contentsname}{目\quad 录}
\newcommand{\nuaa@listfigurename}{插图索引}
\newcommand{\nuaa@listtablename}{表格索引}
\newcommand{\nuaa@listalgorithmname}{表格索引}
\newcommand{\nuaa@bibname}{参考文献}
\newcommand{\nuaa@acknowledgementtitle}{致\quad 谢}
\newcommand{\nuaa@acknowledgementtitletoc}{致谢}
\newcommand{\nuaa@label@algo}{算法}
\newcommand{\nuaa@value@listingname}{代码}

%%==========================
%% labels in the title page
%%==========================
\newcommand\@nuaa{南京航空航天大学}
\newcommand\nuaa@coverpagetitle{毕~业~论~文}
\newcommand\nuaa@bachgrouptitle{毕业设计(论文)团队总报告}
\newcommand\nuaa@label@thesisnum{编号}
\newcommand\nuaa@label@title{题\quad 目}
\newcommand\nuaa@label@teamname{团队名称}
\newcommand\nuaa@label@author{学生姓名}
\newcommand\nuaa@label@studentid{学  号}
\newcommand\nuaa@label@college{学  院}
\newcommand\nuaa@label@major{专  业}
\newcommand\nuaa@label@classid{班  级}
\newcommand\nuaa@label@advisor{指导教师}

%%==============================
%% the theorem name definitions
%%==============================
\def\nuaa@label@algo{算法}
\def\nuaa@label@thm{定理}
\def\nuaa@label@lem{引理}
\def\nuaa@label@prop{命题}
\def\nuaa@label@cor{推论}
\def\nuaa@label@defn{定义}
\def\nuaa@label@conj{猜想}
\def\nuaa@label@exmp{例}
\def\nuaa@label@rem{注}
\def\nuaa@label@case{情形}
\def\nuaa@label@proof{证明}

%%==============================
%% the paper&design setting
%%==============================
\ifnuaa@papertype@design
  \def\nuaa@value@papertype{设计}
\else
  \ifnuaa@papertype@paper
    \def\nuaa@value@papertype{论文}
  \fi
\fi

\endinput
%%
%% End of file `nuaathesis.cfg'.
}
%    \end{macrocode}
%
% \subsubsection{合法性检查与常量定义}
% 必须指定用户类型
%    \begin{macrocode}
\ifnuaa@bachelor\relax\else
  \ifnuaa@master\relax\else
    \ifnuaa@doctor\relax\else
      \ClassError{nuaathesis}{
        Thesis Degree must be specified: \MessageBreak
        degree=[bachelor|master|doctor]}
    \fi
  \fi
\fi
%    \end{macrocode}
% 本科生必须指定文档类型;硕士、博士默认为论文,且必须选择论文。
%    \begin{macrocode}
\ifnuaa@bachelor
  \ifnuaa@worktype@paper\relax\else
    \ifnuaa@worktype@design\relax\else
      \ClassError{nuaathesis}{Thesis Type must be specified: \MessageBreak
      type=[paper|design]}
    \fi
  \fi
\else
  \ifnuaa@worktype@design
    \ClassError{nuaathesis}{You should submit paper instead of design}
  \else
    \nuaa@worktype@papertrue
  \fi
\fi
%    \end{macrocode}
% 默认论文的主语言是中文。
%    \begin{macrocode}
\ifnuaa@lang@cn\relax\else
  \ifnuaa@lang@en\relax\else
    \ifnuaa@lang@ja\relax\else
      \nuaa@lang@cntrue
    \fi
  \fi
\fi
%    \end{macrocode}
% 根据学校信息,定义对应图标。
%    \begin{macrocode}
\ifnuaa@jincheng
  \newcommand\nuaa@university{\nuaa@label@nuaajc}
  \newcommand\nuaa@universityLogo{nuaa-jc.jpg}
\else
  \newcommand\nuaa@university{\nuaa@label@nuaa}
  \newcommand\nuaa@universityLogo{nuaa.pdf}
\fi
%    \end{macrocode}
% 根据文档类型,设置文档的中文名称。
%    \begin{macrocode}
\newcommand\nuaa@worktypecn{
  \ifnuaa@bachelor
    \ifnuaa@worktype@paper
      \nuaa@label@worktype@paper
    \else
      \nuaa@label@worktype@design
    \fi
  \else
    \ifnuaa@master
      \nuaa@label@worktype@master
    \else
      \nuaa@label@worktype@doctor
    \fi
  \fi
}
%    \end{macrocode}
% \subsubsection{文档信息收集}
%    \begin{macrocode}
\def\nuaaset{\kvsetkeys{nuaa@value}}
\def\nuaasetEn{\kvsetkeys{nuaa@valueEn}}
\def\nuaa@def@term #1{
  \define@key{nuaa}{#1}{\csname #1\endcsname{##1}}
  \expandafter\gdef\csname #1\endcsname##1{
    \expandafter\gdef\csname nuaa@#1\endcsname{##1}}
  \csname #1\endcsname{}
}
\def\nuaa@parse@keywords#1#2{
  \define@key{nuaa}{#1}{\csname #1\endcsname{##1}}
  \expandafter\gdef\csname nuaa@#1\endcsname{}
  \expandafter\gdef\csname #1\endcsname##1{
    \@for\reserved@a:=##1\do{
      \expandafter\ifx\csname nuaa@#1\endcsname\@empty\else
        \expandafter\g@addto@macro\csname nuaa@#1\endcsname{%
          \ignorespaces #2}
      \fi
      \expandafter\expandafter\expandafter\g@addto@macro%
        \expandafter\csname nuaa@#1\expandafter\endcsname\expandafter{\reserved@a}}}}
%    \end{macrocode}
%
% 文档的中文信息
%    \begin{macrocode}
\nuaa@def@term{value@title}
\nuaa@def@term{value@author}
\nuaa@def@term{value@studentid}
\nuaa@def@term{value@college}
\nuaa@def@term{value@major}
\nuaa@def@term{value@classid}
\nuaa@def@term{value@advisors}
\nuaa@def@term{value@applydate}
\nuaa@def@term{value@libraryclassid}
\nuaa@def@term{value@subjectclassid}
\nuaa@def@term{value@thesisid}
\nuaa@def@term{value@majorsubject}
\nuaa@def@term{value@researchfield}
%    \end{macrocode}
%
% 文档的英文信息
%    \begin{macrocode}
\nuaa@def@term{valueEn@title}
\nuaa@def@term{valueEn@college}
\nuaa@def@term{valueEn@majorsubject}
\nuaa@def@term{valueEn@author}
\nuaa@def@term{valueEn@advisors}
\nuaa@def@term{valueEn@applydate}
%    \end{macrocode}
%
% 摘要页
%    \begin{macrocode}
\newcommand{\nuaa@@abstract}[1]{\long\gdef\nuaa@abstract{#1}}
\newenvironment{abstract}{\Collect@Body\nuaa@@abstract}{}
\newcommand{\nuaa@@abstractEn}[1]{\long\gdef\nuaa@abstractEn{#1}}
\newenvironment{abstractEn}{\Collect@Body\nuaa@@abstractEn}{}
\nuaa@parse@keywords{keywords}{\nuaa@label@keywordsep}
\nuaa@parse@keywords{keywordsEn}{\nuaa@labelEn@keywordsep}
%    \end{macrocode}
%
% 收集一些常用字段
%    \begin{macrocode}
\ifnuaa@lang@cn
  \newcommand\nuaa@title{\nuaa@value@title}
\else\ifnuaa@lang@en
  \newcommand\nuaa@title{\nuaa@valueEn@title}
\fi\fi
%    \end{macrocode}
% \subsection{载入宏包}
% \subsubsection{主文档类 \texttt{ctexbook}}
%
% 根据要求,本科生论文字号为小四(12~pt),行间距为1.5~倍。
% 考虑到学校提供的 Word 模板启用了文档网络,跨度为 15.6~pt,所以小四号字只占用一行,
% 行间距需修正为 $15.6 \times 1.5 = 23.4$~pt\footnote{\url{https://www.zhihu.com/question/26397264/answer/48165229}}。
% 再考虑到用 \verb|\zihao| 在选择字号时,\pkg{ctex} 会将 \verb|\f@baselineskip|
% 设置成 \verb|\f@size| 的 1.2 倍\footnote{\url{https://liam0205.me/2013/10/17/LaTeX-Linespace/}},
% 所以最终设置行间距为 $23.4 \div (12 \times 1.2) = 1.625$。
%
% 同理,硕/博士论文字号为五号(10.5~pt),行间距为20~pt。
% 因为行间距是固定值,所以 Word 的文档网络不会影响行间距。
% 行间距最终设置为 $20 \div (10.5 \times 1.2) \approx 1.5873$。
%    \begin{macrocode}
\ifnuaa@bachelor
  \PassOptionsToClass{zihao=-4,linespread=1.625}{ctexbook}
\else
  \PassOptionsToClass{zihao=5,linespread=1.5873}{ctexbook}
\fi
\PassOptionsToClass{a4paper,scheme=chinese,space=auto,UTF8}{ctexbook}
\LoadClass{ctexbook}
%    \end{macrocode}
%
% \subsubsection{\LaTeX 增强}
%    \begin{macrocode}
\RequirePackage{etoolbox}
%    \end{macrocode}
%
% \subsubsection{数学相关}
% 尽早引入这些宏包,避免载入顺序引发的问题。
%    \begin{macrocode}
\RequirePackage{amsmath,amsthm}
%    \end{macrocode}
%
% \subsubsection{其他?}
% 首段缩进
%    \begin{macrocode}
\RequirePackage{indentfirst}
%    \end{macrocode}
% 硕/博士封面一开始需要分栏
%    \begin{macrocode}
\RequirePackage{multicol}
%    \end{macrocode}
% 一些 \TeX 系列的 logo
%    \begin{macrocode}
\RequirePackage{hologo}
%    \end{macrocode}
% Todo: 用途不明,求文档
%    \begin{macrocode}
\PassOptionsToPackage{no-math}{footspec}
\RequirePackage{fontspec}
\RequirePackage[perpage,bottom]{footmisc}
\RequirePackage{soul}
\RequirePackage{xltxtra}
%    \end{macrocode}
%
% \subsection{格式设置}
% \subsubsection{英文字体 \pkg{newtxtext}}
%
% \pkg{newtxtext} 与 Monotype 的 Times New Roman\textsuperscript{\textregistered} 有一点点差别。
% \pkg{newtxtext} 收录于CTAN,而大部分 Linux 因为版权限制,不自带 Times New Roman\textsuperscript{\textregistered} 字体。
% 由于两者差别很小,优先使用 \pkg{newtxtext} 提供的英文字体。
%    \begin{macrocode}
\RequirePackage{newtxtext}
%    \end{macrocode}
%
% \subsubsection{页边距 \pkg{geometry}}
%
% 本科生的页边距沿用前一版本的设定(Word 在页眉里插入图片,行高不知道怎么算)。
%
% 硕/博士的页边距是根据要求计算出来的。
% \texttt{headheight}和\texttt{footskip}是五号字的行高(考虑文档网络),
% $\texttt{headsep} = 3.3\text{cm}-2.6\text{cm}-15.6\text{pt} \approx 0.15\text{cm}$。
%    \begin{macrocode}
\RequirePackage{geometry}
\ifnuaa@bachelor \geometry{
  top=2.5cm,
  bottom=2cm,
  left=2cm,
  right=2cm,
  headheight=0.75cm,
  headsep=2bp,
  %footskip=0.8cm,
  includehead,
  includefoot
} \else \geometry{
  top=3.3cm,
  bottom=3.3cm,
  left=3.0cm,
  right=2.6cm,
  headheight=15.6bp,
  headsep=0.15cm,
  footskip=15.6bp
}
\fi
%    \end{macrocode}
%
% \subsubsection{页眉页脚 \pkg{fancyhdr}}
%    \begin{macrocode}
\RequirePackage{fancyhdr}
%    \end{macrocode}
%
% 页码修饰,本科生正文和附录部分的页码格式比较特别,目录里的页码格式与页脚页码格式不同,
% 所以页脚页码需要特殊处理。
%    \begin{macrocode}
\newcommand\nuaa@pagenum@decorate[1]{
  \ifnuaa@bachelor
    \if@frontmatter
      #1
    \else
      - #1 -
    \fi
  \else
    #1
  \fi
}
%    \end{macrocode}
%
% 定义不带页眉页脚的空白样式,用于封面页和空白页。
%    \begin{macrocode}
\fancypagestyle{style@empty}{
  \fancyhf{}
  \renewcommand{\headrulewidth}{0pt}
  \renewcommand{\footrulewidth}{0pt}
}
%    \end{macrocode}
%
% 定义正文页眉页脚的样式。
%    \begin{macrocode}
\fancypagestyle{style@main}{
  \fancyhead{}
  \ifnuaa@bachelor
    \fancyhead[L]{
      \setlength{\unitlength}{1mm}
      \begin{picture}(0,0)
        \put(7.3,1.5){\includegraphics[width=6cm]{\nuaa@universityLogo}}
      \end{picture}
    }
    \fancyhead[R]{\songti\zihao{4}\nuaa@worktypecn \nuaa@label@reportpaper\hspace{1\ccwd}}
  \else
    \fancyhead[C]{
      \songti\zihao{5}
      \ifthenelse{\isodd{\value{page}}}{
        \nuaa@university\nuaa@worktypecn
      }{
        \nuaa@title
      }}
  \fi
  \fancyfoot{}
  \if@twoside
    \fancyfoot[EL]{\footnotesize{\nuaa@pagenum@decorate{\thepage}}}
    \fancyfoot[OR]{\footnotesize{\nuaa@pagenum@decorate{\thepage}}}
  \else
    \fancyfoot[R]{\footnotesize{\nuaa@pagenum@decorate{\thepage}}}
  \fi
  \renewcommand{\headrulewidth}{0.75bp}
  \ifnuaa@bachelor
    \renewcommand{\footrulewidth}{0.75bp}
  \fi
}
%    \end{macrocode}
% \subsubsection{front/main/backmatter}
% 重定义了\texttt{\textbackslash frontmatter}、\texttt{\textbackslash mainmatter}和\texttt{\textbackslash backmatter}
% 三个宏,用来调整文档不同部分的格式。
%    \begin{macrocode}
\newif\if@frontmatter
\newif\if@backmatter
\renewcommand{\frontmatter}{
  \cleardoublepage
  \@frontmattertrue
  \@backmatterfalse
  \ifnuaa@bachelor\pagenumbering{roman}\else\pagenumbering{Roman}\fi
  \pagestyle{style@main}
  \ctexset{chapter = {
    break = {\clearpage},
    numbering = false
  }}
}
\renewcommand{\mainmatter}{
  \cleardoublepage
  \@frontmatterfalse
  \@backmatterfalse
  \pagenumbering{arabic}
  \pagestyle{style@main}
  \ctexset{chapter = {
    break = {\if@openright\cleardoublepage\else\clearpage\fi},
    numbering = true
  }}
}
\renewcommand{\backmatter}{
  \cleardoublepage
  \@frontmatterfalse
  \@backmattertrue
  \ctexset{chapter/numbering = false,
    section/number = \Alph{section},
    section/name = {,.},
    subsection/number = \CTEXthesection{}\arabic{subsection},}%
  \setcounter{chapter}{0}
}
%    \end{macrocode}
% \subsubsection{目录 \pkg{tocloft}}
% 设置目录的缩进、字体(通常)。
%    \begin{macrocode}
\RequirePackage[]{tocloft}
\setcounter{secnumdepth}{3}  %% 章节编号深度 (part 对应 -1)
\setcounter{tocdepth}{2}     %% 目录深度 (part 对应 -1)
\setlength{\cftbeforechapskip}{0bp}
\renewcommand{\cftchapleader}{\cftdotfill{\cftdotsep}}%% 目录中章后也有点
\tocloftpagestyle{style@main}
\newcommand{\tocfont}{\normalsize}
\renewcommand{\cftchapfont}{\tocfont}
\renewcommand{\cftsecfont}{\tocfont}
\renewcommand{\cftsubsecfont}{\tocfont}
\renewcommand{\cftsubsubsecfont}{\tocfont}
\renewcommand{\cftfigfont}{\tocfont}
\renewcommand{\cfttabfont}{\tocfont}
\renewcommand{\cftchappagefont}{\tocfont}
\renewcommand{\cftsecpagefont}{\tocfont}
\renewcommand{\cftsubsecpagefont}{\tocfont}
\renewcommand{\cftsubsubsecpagefont}{\tocfont}
\renewcommand{\cftfigpagefont}{\tocfont}
\renewcommand{\cfttabpagefont}{\tocfont}
%    \end{macrocode}
%
% 进一步设置目录格式(非常规手段),去除 \pkg{tocloft} 绘制标题的代码。
%    \begin{macrocode}
\renewcommand{\@cftmaketoctitle}{}
\renewcommand{\@cftmakeloftitle}{}
\renewcommand{\@cftmakelottitle}{}
%    \end{macrocode}
%
% 绘制正文目录,本科生“目录”字号(三号)与其他大标题(chapter,小三)不同。
%    \begin{macrocode}
\newcommand\nuaatableofcontents{
  \cleardoublepage
  \chapter*{
    \ifnuaa@bachelor
      \linespread{1.5}\fontsize{16bp}{15.6bp}\selectfont
    \fi
    \nuaa@label@tableofcontents}
  \tableofcontents
}
%    \end{macrocode}
%
% 绘制图表清单,两张不带标题的表合并在一起,每一条添加“图”、“表”前缀。
%    \begin{macrocode}
\newcommand\nuaalistoffigurestables{
  \clearpage
  \renewcommand{\cftfigpresnum}{\nuaa@label@figurename}
  \renewcommand{\cfttabpresnum}{\nuaa@label@tablename}
  \newlength{\mylenf}
  \settowidth{\mylenf}{\cftfigpresnum}
  \setlength{\cftfignumwidth}{\dimexpr\mylenf+2em}
  \setlength{\cfttabnumwidth}{\dimexpr\mylenf+2em}

  \chapter*{\nuaa@label@listoffigurestables}
  \listoffigures
  \listoftables
}
%    \end{macrocode}
% \subsubsection{各级标题}
% 设置各级标题的行间距、字体、字号等。
%
% 由于 chapter 级标题在页首时,会引入一些空白,暂时没找到这些数值是怎么计算的,
% 所以根据实际测量,修正 \texttt{beforeskip} 的值。
%
% \pkg{ctex} 提供的参数 \texttt{fixskip} 会把正常的行间距一并删减掉,
% 修正这些误差的操作更复杂,所以放弃不用。
%    \begin{macrocode}
\ifnuaa@bachelor \ctexset{
  chapter = {
    pagestyle = style@main,
    format = {\centering\linespread{2.41}\heiti\fontsize{15bp}{15.6bp}\selectfont},
    beforeskip = {23.7bp},  % 15.6*2 + 17 - 24.5(correction) = 35.2bp
    afterskip = {16.5bp},
  },
  section = {
    format = {\linespread{1.5}\heiti\fontsize{14bp}{20.8bp}\selectfont},
    beforeskip = {0bp},
    afterskip = {0bp},
  },
  subsection = {
    format = {\linespread{1.5}\heiti\fontsize{12bp}{15.6bp}\selectfont},
    indent = {0\ccwd},
    beforeskip = {0bp},
    afterskip = {0bp},
  },
  subsubsection = {
    format = {\heiti\fontsize{12bp}{20bp}},
    indent = {0\ccwd},
    beforeskip = {7.8bp},
    afterskip = {7.8bp},
  }
} \else \ctexset{
  chapter = {
    pagestyle = style@main,
    format = {\centering\linespread{1.0}\heiti\fontsize{15bp}{20bp}\selectfont},
    beforeskip = {10.4bp},   % 15.6*1.5 - 13(correction) = 10.4
    afterskip = {23.4bp},
    },
  section = {
    format = {\linespread{1.0}\heiti\fontsize{14bp}{20bp}\selectfont},
    beforeskip = {7.8bp},
    afterskip = {7.8bp},
  },
  subsection = {
    format = {\linespread{1.0}\heiti\fontsize{12bp}{20bp}\selectfont},
    beforeskip = {7.8bp},
    afterskip = {7.8bp},
    indent = {0\ccwd},
  },
  subsubsection = {
    format = {\linespread{1.0}\heiti\fontsize{12bp}{20bp}},
    indent = {0\ccwd},
    beforeskip = {7.8bp},
    afterskip = {7.8bp},
  }
}
\fi
%    \end{macrocode}
% \subsubsection{参考文献 \pkg{natbib}}
% 调整字体大小和行间距,本/硕/博貌似都是用五号字、单倍行距(文档网络修正行间距为15.6~pt)。
%    \begin{macrocode}
\RequirePackage[numbers,square,comma,super,sort&compress]{natbib}
\AtBeginDocument{
  \pretocmd{\bibliography}{\begingroup\linespread{1.0}\fontsize{10.5bp}{15.6bp}\selectfont}{}{}
  \apptocmd{\bibliography}{\endgroup}{}{}
}
\renewcommand\@biblabel[1]{\linespread{1.0}\fontsize{10.5bp}{15.6bp}\selectfont[#1]}
\renewenvironment{thebibliography}[1]{
  \chapter*{\bibname}
  \list{\@biblabel{\@arabic\c@enumiv}}
  {
    \settowidth\labelwidth{\@biblabel{#1}}
    \leftmargin\labelwidth
    \advance\leftmargin\labelsep
    \setlength{\parsep}{1mm}
    \setlength{\labelsep}{0.5em}
    \setlength{\itemsep}{0.05pc}
    \setlength{\listparindent}{0in}
    \setlength{\itemindent}{0in}
    \setlength{\rightmargin}{0in}
    \@openbib@code
    \usecounter{enumiv}
    \let\p@enumiv\@empty
    \renewcommand\theenumiv{\@arabic\c@enumiv}
  }
  \sloppy
  \clubpenalty4000
  \@clubpenalty\clubpenalty
  \widowpenalty4000%
  \sfcode`\.\@m
}{
\def\@noitemerr
{\@latex@warning{Empty `thebibliography' environment}}
\endlist \vskip.2in}
%    \end{macrocode}
%
% 将“参考文献”加入目录和 pdf 书签。
%    \begin{macrocode}
\pretocmd{\bibliography}{
  \clearpage\phantomsection\addcontentsline{toc}{chapter}{\bibname}}{}{}
%    \end{macrocode}
%
% 另外的引用命令。
%    \begin{macrocode}
\bibpunct{[}{]}{,}{s}{}{,}
\renewcommand\NAT@citesuper[3]{\ifNAT@swa%
  \unskip\kern\p@\textsuperscript{\NAT@@open #1\NAT@@close}%
  \if*#3*\else\ (#3)\fi\else #1\fi\endgroup}
\DeclareRobustCommand\inlinecite{\@inlinecite}
\def\@inlinecite#1{\begingroup\let\@cite\NAT@citenum\citep{#1}\endgroup}
\let\onlinecite\inlinecite
%    \end{macrocode}
% \subsection{文档部件}
% \subsubsection{定理环境}
%    \begin{macrocode}
\newtheoremstyle{break}% name
   {}%      Space above, empty = `usual value'
   {}%      Space below
   {\itshape}% Body font
   {}%         Indent amount (empty = no indent, \parindent = para indent)
   {\bfseries}% Thm head font
   {.}%        Punctuation after thm head
   {\newline}% Space after thm head: \newline = linebreak
   {}%         Thm head spec
%</cls>
%<*cfg>
%%%%% theorem %%%%%
\ifnuaa@lang@cn
\theoremstyle{plain}
  \newtheorem{algo}{算法~}[chapter]
\theoremstyle{definition}
  \newtheorem{assumption}{假设~}[chapter]
  \newtheorem{thm}{定理~}[chapter]
  \newtheorem{defn}{定义~}[chapter]
  \newtheorem{conj}{猜想~}[chapter]
  \newtheorem{exmp}{例~}[chapter]
  \newtheorem{rem}{注~}
  \newtheorem{case}{情形~}
  \newtheorem{lem}[thm]{引理~}
  \newtheorem{prop}[thm]{命题~}
  \newtheorem{cor}[thm]{推论~}
\theoremstyle{break}
  \newtheorem{bthm}[thm]{定理~}
  \newtheorem{blem}[thm]{引理~}
  \newtheorem{bprop}[thm]{命题~}
  \newtheorem{bcor}[thm]{推论~}
  \renewcommand{\proofname}{\bf 证明}
\else
  \ClassError{nuaathesis}{Internal error}
\fi
%</cfg>
%    \end{macrocode}
% \subsubsection{列表环境}
%    \begin{macrocode}
%<*cls>
\RequirePackage[inline]{enumitem}
%    \end{macrocode}
% 使用enumitem宏包配制列表环境,紧凑间距。
%    \begin{macrocode}
\setlist{nosep}
%    \end{macrocode}
% 列表和段落头对齐
%    \begin{macrocode}
\setlist*{leftmargin=*}
\setlist[1]{labelindent=\parindent} %% Only the level 1
%    \end{macrocode}
% \subsubsection{图表和题注}
% 图
%    \begin{macrocode}
\RequirePackage{graphicx} %% 插图相关
\graphicspath{{fig/}{figure/}{figures/}{logo/}{logos/}{graph/}{graphs}}
\DeclareGraphicsExtensions{.pdf,.eps,.png,.jpg,.jpeg}
%    \end{macrocode}
%
% 表
%    \begin{macrocode}
\RequirePackage{booktabs}
\RequirePackage{longtable}
\RequirePackage{makecell}
\RequirePackage{multirow}
\RequirePackage{rotating}
\RequirePackage{diagbox}
\RequirePackage{tabularx}
\RequirePackage[table]{xcolor}
%    \end{macrocode}
%
% 题注
%    \begin{macrocode}
\RequirePackage{caption}
\RequirePackage{subcaption}
\RequirePackage{footnote}
%    \end{macrocode}
% 题注:在图表内使用脚注
%    \begin{macrocode}
\makesavenoteenv{tabular}
%    \end{macrocode}
% 题注格式:编号与题注间隔,最后一行居中,本科题注为黑体
%    \begin{macrocode}
\captionsetup{labelsep=quad}
\captionsetup{justification=centerlast}
\ifnuaa@bachelor
  \DeclareCaptionFont{heiticaption}{
    \linespread{1.5}\heiti\fontsize{10.5bp}{15.6bp}\selectfont}
  \captionsetup{font=heiticaption}
\fi
%    \end{macrocode}
% \subsubsection{双语题注}
% 利用\pkg{newfloat} 创建一些 float 环境,将第二语言的题注声明在其他 float 环境内,
% 避免图表目录里同时出现一张图/表的两个不同语言的名字。
%    \begin{macrocode}
\RequirePackage{float}
\RequirePackage{newfloat}
\RequirePackage[labelformat=simple]{subcaption}
\RequirePackage{bicaption}
\captionsetup[bi-first]{bi-first}
\captionsetup[bi-second]{bi-second}
%</cls>
%<*cfg>
%%%%% bi-caption %%%%%
\ifnuaa@lang@cn
\DeclareCaptionOption{bi-first}[]{
  \def\tablename{\nuaa@label@tablename}
  \def\figurename{\nuaa@label@figurename}
}
\DeclareCaptionOption{bi-second}[]{
  \def\tablename{\nuaa@labelEn@tablename}
  \def\figurename{\nuaa@labelEn@figurename}
}
\DeclareFloatingEnvironment[fileext=lofEN]{figuresEN}[\shortfigurenameEN][\listoffiguresnameEN]
\DeclareFloatingEnvironment[fileext=lotEN]{tablesEN}[\shorttablenameEN][\listoftablesnameEN]
\captionsetup[bi-second]{listtype+=sEN}
\else
  \ClassError{nuaathesis}{Internal error}
\fi
%</cfg>
%    \end{macrocode}
%
% \subsubsection{自动交叉引用}
%    \begin{macrocode}
%<*cls>
\RequirePackage[hidelinks]{hyperref}
%    \end{macrocode}
% 顺便给 PDF 书签加上自动编号,\pkg{bookmark}需要\pkg{hyperref}后载入。
%    \begin{macrocode}
\RequirePackage[numbered]{bookmark}
%</cls>
%<*cfg>
%%%%% auto ref %%%%%
\ifnuaa@lang@cn
\AtBeginDocument{%
  \def\figureautorefname{图}
  \def\tableautorefname{表}
  \def\partautorefname{卷}
  \def\appendixautorefname{附录}
  \def\equationautorefname{式}
  \def\Itemautorefname{列表}
  \def\chapterautorefname{章}
  \def\sectionautorefname{节}
  \def\subsectionautorefname{小节}
  \def\subsubsectionautorefname{条目}
  \def\paragraphautorefname{自然段}
  \def\Hfootnoteautorefname{脚注}
  \def\AMSautorefname{式}
  \def\theoremautorefname{理论}
  \def\pageautorefname{页}
}
\else
  \ClassError{nuaathesis}{Internal error}
\fi
%</cfg>
%    \end{macrocode}
% \subsubsection{代码清单 \pkg{listing}}
%    \begin{macrocode}
%<*cls>
\RequirePackage{listings}
%</cls>
%<*cfg>
%%%%% listings %%%%%
\renewcommand\lstlistingname{\nuaa@label@lstlistingname}
\lstset{tabsize=4, %
  frame=shadowbox, %把代码用带有阴影的框圈起来
  commentstyle=\color{red!50!green!50!blue!50},%浅灰色的注释
  rulesepcolor=\color{red!20!green!20!blue!20},%代码块边框为淡青色
  keywordstyle=\color{blue!90}\bfseries, %代码关键字的颜色为蓝色,粗体
  showstringspaces=false,%不显示代码字符串中间的空格标记
  stringstyle=\ttfamily, % 代码字符串的特殊格式
  keepspaces=true, %
  breakindent=22pt, %
  numbers=left,%左侧显示行号
  stepnumber=1,%
  numberstyle=\tiny, %行号字体用小号
  basicstyle={\footnotesize\ttfamily}, %
  showspaces=false, %
  flexiblecolumns=true, %
  breaklines=true, %对过长的代码自动换行
  breakautoindent=true,%
  breakindent=4em, %
  aboveskip=1em, %代码块边框
  %% added by http://bbs.ctex.org/viewthread.php?tid=53451
  fontadjust,%
  captionpos=t,%
  framextopmargin=2pt,framexbottommargin=2pt,abovecaptionskip=-3pt,belowcaptionskip=3pt,%
  xleftmargin=4em,xrightmargin=4em, % 设定listing左右的空白
  texcl=true,
  % 设定中文冲突,断行,列模式,数学环境输入,listing数字的样式
  extendedchars=false,columns=flexible,mathescape=true%
  numbersep=-1em%
}
%</cfg>
%    \end{macrocode}
% \subsection{实用命令}
% \subsubsection{中文日期}
%    \begin{macrocode}
%<*cls>
\newcommand{\nuaa@dateCn}{
  \zhdigits{\the\year}年\zhnumber{\the\month}月
}
%    \end{macrocode}
%
% \subsubsection{英文日期}
%    \begin{macrocode}
\newcommand{\nuaa@dateEn}{
  \ifcase\the\month
  \or January%
  \or February%
  \or March%
  \or April%
  \or May%
  \or June%
  \or July%
  \or August%
  \or September%
  \or October%
  \or November%
  \or December%
  \fi, \the\year
}
%    \end{macrocode}
%
% \subsubsection{盲审信息抹除}
%    \begin{macrocode}
\newcommand\secretize[1]{
  \ifnuaa@bindtrail
    ***
  \else
    #1
  \fi
}
%    \end{macrocode}
%
% \subsubsection{双面换页}
% 为了产生真正的空白页,需要手动结束当前页,设置页眉页脚,然后再双面换页。
%    \begin{macrocode}
\let\nuaa@cleardoublepage\cleardoublepage
\renewcommand{\cleardoublepage}{
  \clearpage
  {
  \ifnuaa@blankleft\pagestyle{style@empty}\fi
  \nuaa@cleardoublepage
  }
}
%    \end{macrocode}
%
% \subsubsection{论文模板的logo}
%    \begin{macrocode}
\newcommand{\nuaathesis}{%
  \makebox{%
    N\hspace{-0.3ex}\raisebox{-0.5ex}{U}\hspace{-0.3ex}A\mbox{$^{\hspace{-0.5ex}2}$}\hspace{0.3ex}%
    \textsc{Thesis}}}
\newcommand{\oldnuaathesis}{%
  N\raisebox{0.5ex}{U}\hspace{-0.3ex}AA%
  \textsc{Thesis}
}
\newcommand{\seuthesix}{%
  \makebox{S\hspace{-0.3ex}\raisebox{-0.5ex}{E}\hspace{-0.3ex}U\hspace{0.1em}%
  \textsc{Thesix}}
}
%    \end{macrocode}
%
% \subsubsection{绘制关键词}
% 摘要页的关键词部分,如果关键词换行了,需要与第一个关键词左对齐。
%    \begin{macrocode}
\newbox\nuaa@kw
\newcommand{\nuaa@put@kw}[2]{%
  \begingroup
  \setbox\nuaa@kw=\hbox{#1}
  \noindent\hangindent\wd\nuaa@kw\hangafter1
  \box\nuaa@kw#2\par
  \endgroup}
%    \end{macrocode}
%
% \subsubsection{特殊页面}
% 按照文档的类别,调用实际的宏。
%    \begin{macrocode}
\def\makecover{
  \hypersetup{
    % TODO: language
    pdftitle = {\nuaa@value@title},
    pdfauthor = {\nuaa@value@author},
    pdfsubject = {\nuaa@worktypecn},
    pdfkeywords = {\nuaa@keywords},
    % pdfcreator = {my logo here?}
  }

  \pagestyle{style@empty}
  \pagenumbering{Alph}
  \cleardoublepage

  \phantomsection
  \ifnuaa@bachelor
    \nuaa@make@cover@bachelor
    \relax
  \else
    \nuaa@make@cover@master@cn
    \nuaa@make@cover@master@en
  \fi
}

\def\makedeclare{
  \ifnuaa@bachelor
    \nuaa@make@declare@bachelor
  \else
    \nuaa@make@declare@master
  \fi
}

\def\makeabstract{
  \ifnuaa@bachelor
    \nuaa@make@abstract@bachelor@cn
    \nuaa@make@abstract@bachelor@en
  \else
    \nuaa@make@abstract@master@cn
    \nuaa@make@abstract@master@en
  \fi
}
%    \end{macrocode}
%
% \subsection{绘制特殊页面}
% \subsubsection{本科封面}
%    \begin{macrocode}
\newcommand{\nuaa@make@cover@bachelor}{
  \cleardoublepage
  \newgeometry{top=1.0in, bottom=1.0in, left=1.25in, right=1.25in}

  \begingroup
  \linespread{1.0}

  \begin{flushright}
    \heiti\fontsize{14bp}{21bp}\nuaa@label@thesisnum
    \underline{\hspace{60bp}}
    \\ \vspace{42bp}
  \end{flushright}

  \begin{center}
    \textbf{\kaishu\fontsize{22bp}{30bp}\nuaa@university}
    \\ \vspace{30bp}

    \textbf{
      \songti\fontsize{55pt}{32pt}\selectfont
      \xeCJKsetup{CJKglue={\hskip 10pt plus 0.08\baselineskip}}
      \nuaa@worktypecn}
    \\ \vspace{3bp}       % 56bp(origin) + 7bp(correction) - 60bp(merged to below)

    \begingroup
    \heiti\zihao{2}
    \parbox[c][142bp][c]{1.44in}{\center\nuaa@label@title}
    \parbox[c][142bp][c]{4.07in}{\center
      \ifnuaa@lang@cn\relax\else
      \nuaa@valueEn@title \par
      \fi
      \nuaa@value@title}
    \endgroup

    \begingroup
    \renewcommand{\tabcolsep}{0bp}
    \renewcommand{\arraystretch}{2.23}
    \newcommand\nuaa@make@cover@label[1]{
      \makebox[1.31in][c]{\heiti\zihao{-3}\makebox[4\ccwd][s]{##1}}}
    \newcommand\nuaa@make@cover@field[1]{
      \makebox[3.22in][c]{\heiti\zihao{3}\secretize{##1}}}
    \begin{tabular}{cc}
      \nuaa@make@cover@label{\nuaa@label@author} &
      \nuaa@make@cover@field{\nuaa@value@author} \\ \cline{2-2}
      \nuaa@make@cover@label{\nuaa@label@studentid} &
      \nuaa@make@cover@field{\nuaa@value@studentid} \\ \cline{2-2}
      \nuaa@make@cover@label{\ifnuaa@jincheng\nuaa@label@department\else\nuaa@label@college\fi} &
      \nuaa@make@cover@field{\nuaa@value@college} \\ \cline{2-2}
      \nuaa@make@cover@label{\nuaa@label@major} &
      \nuaa@make@cover@field{\nuaa@value@major} \\ \cline{2-2}
      \nuaa@make@cover@label{\nuaa@label@classid} &
      \nuaa@make@cover@field{\nuaa@value@classid} \\ \cline{2-2}
      \nuaa@make@cover@label{\nuaa@label@advisor} &
      \nuaa@make@cover@field{\nuaa@value@advisors} \\ \cline{2-2}
    \end{tabular}
    \endgroup
    \vspace{43bp}

    \heiti\zihao{3}
    \ifdefempty{\nuaa@value@applydate}{\nuaa@dateCn}{\nuaa@value@applydate}
  \end{center}

  \endgroup
  \restoregeometry
}
%    \end{macrocode}
% \subsubsection{本科承诺书}
%    \begin{macrocode}
\newcommand{\nuaa@make@declare@bachelor}{
  \cleardoublepage
  \newgeometry{top=1.0in, bottom=1.0in, left=1.25in, right=1.25in}

  \begin{center}
    \linespread{1.0}\heiti\fontsize{18bp}{31.2bp}\selectfont
    \textbf{\nuaa@university} \par
    \textbf{本科\nuaa@worktypecn 诚信承诺书}
  \end{center}

  \begingroup
    \linespread{1.0}\songti\fontsize{14bp}{31.2bp}\selectfont
    本人郑重声明:所呈交的\nuaa@worktypecn
    (题目:\uline{\nuaa@title} )
    是本人在导师的指导下独立进行研究所取得的成果。
    尽本人所知,除了\nuaa@worktypecn 中特别加以标注引用的内容外,
    本\nuaa@worktypecn 不包含任何其他个人或集体已经发表或撰写的成果作品。

    \vspace{31.2bp}

    \begin{flushright}
      \setlength{\tabcolsep}{0bp}
      \begin{tabular}{rcr}
      作者签名: & \hspace{7.5em} & \hspace{2em} 年 \hspace{0.5em} 月 \hspace{0.5em} 日 \\
      (学号): & \hspace{7.5em} & \\
      \end{tabular}
    \end{flushright}

  \endgroup
  \restoregeometry
}
%    \end{macrocode}
% \subsubsection{本科中文摘要}
%    \begin{macrocode}
\newcommand{\nuaa@make@abstract@bachelor@cn}{
  \ifnuaa@abstractopenright
    \cleardoublepage
  \else
    \clearpage
  \fi

  \begin{center}
    \vspace*{-4.3pt}\heiti\zihao{2}
    \phantomsection
    \addcontentsline{toc}{chapter}{\nuaa@label@abstractshort}
    \nuaa@value@title
  \end{center}

  \begin{center}
    \heiti\zihao{-3}\vspace{1em}
    \nuaa@label@abstract
  \end{center}

  \begingroup
  \songti\zihao{-4}
  \nuaa@abstract
  \endgroup

  \vspace{3em}

  \nuaa@put@kw{\textbf{\heiti\zihao{-3}\nuaa@label@keywords}}{\nuaa@keywords}
}
%    \end{macrocode}
% \subsubsection{本科英文摘要}
%    \begin{macrocode}
\newcommand{\nuaa@make@abstract@bachelor@en}{
  \ifnuaa@abstractopenright
    \cleardoublepage
  \else
    \clearpage
  \fi

  \begin{center}
    \vspace*{-4.3pt}\heiti\zihao{2}
    \phantomsection
    \addcontentsline{toc}{chapter}{\nuaa@labelEn@abstract}
    \nuaa@valueEn@title
  \end{center}

  \begin{center}
    \heiti\zihao{-3}\vspace{18pt}
    \nuaa@labelEn@abstract
    \vspace{10pt}
  \end{center}

  \begingroup
  \songti\zihao{-4}
  \nuaa@abstractEn
  \endgroup

  \vspace{3em}

  \nuaa@put@kw{\textbf{\zihao{-3}\nuaa@labelEn@KeyWords}}{\nuaa@keywordsEn}
}
%    \end{macrocode}
% \subsubsection{硕/博士中文封面}
%    \begin{macrocode}
\newcommand{\nuaa@make@cover@master@cn}{
  \cleardoublepage
  \begingroup
  \linespread{1.0}

  \vspace*{-4.8bp} \vspace{-\parskip}\vspace{-\topskip}

  \begin{multicols}{2}
    \linespread{1}\songti\fontsize{10.5bp}{15.6bp}\selectfont
    \begin{flushleft}
      中图分类号:\nuaa@value@libraryclassid \par
      学科分类号:\nuaa@value@subjectclassid
    \end{flushleft}
    \columnbreak
    \begin{flushright}
      论文编号:\nuaa@value@thesisid
    \end{flushright}
  \end{multicols}

  \vspace{61bp} \vspace{-\parskip}\vspace{-\baselineskip}

  \begin{center}
    \songti\zihao{0}\nuaa@worktypecn
  \end{center}

  \vspace{\stretch{0.25}}

  \begin{center}
    \linespread{1.5}\heiti\zihao{1}\nuaa@value@title
  \end{center}
  \vfill

  \begingroup
    \linespread{1.0}\songti\fontsize{16bp}{31.2bp}\selectfont
    \newcommand\nuaa@make@cover@label[1]{\makebox[5\ccwd][s]{##1}}
    \parbox[t]{2.875in}{
      \setlength{\parindent}{1.125in}
      \nuaa@make@cover@label{\nuaa@label@researchername} \par
      \nuaa@make@cover@label{\nuaa@label@majorsubject} \par
      \nuaa@make@cover@label{\nuaa@label@researchfield} \par
      \nuaa@make@cover@label{\nuaa@label@advisor} \par
    }
    \parbox[t]{3in}{
      \setlength{\parindent}{0bp}
      \nuaa@value@author \par
      \nuaa@value@majorsubject \par
      \nuaa@value@researchfield \par
      \nuaa@value@advisors \par
    }
  \endgroup

  \vspace{61bp}

  \begin{center}
    \includegraphics{nuaa-jianqi.pdf}

    \kaishu\fontsize{18bp}{31.2bp}\selectfont
    \nuaa@label@graduateschool\quad \nuaa@value@college

    \kaishu\fontsize{16bp}{31.2bp}\selectfont
    \ifdefempty{\nuaa@value@applydate}{\nuaa@dateCn}{\nuaa@value@applydate}

  \end{center}

  \endgroup
}
%    \end{macrocode}
% \subsubsection{硕/博士英文封面}
%    \begin{macrocode}
\newcommand{\nuaa@make@cover@master@en}{
  \cleardoublepage

  \begin{center}
    \linespread{1.8}\zihao{4}
    \nuaa@labelEn@nuaa \\
    \nuaa@labelEn@graduateschool \\
    \nuaa@valueEn@college

    \vfill
    \begingroup
      \linespread{1.5}
      \zihao{2}\textbf{\nuaa@valueEn@title} \par
    \endgroup
    \vfill

    \linespread{2.0}\zihao{4}
    A Thesis in \\
    \nuaa@valueEn@majorsubject \\
    by \\
    \nuaa@valueEn@author \bigskip

    Advised by \\
    \nuaa@valueEn@advisors \bigskip

    Submitted in Partial Fulfillment \\
    of the Requirements \\
    for the Degree of \\
    \ifnuaa@master
      \nuaa@labelEn@degreemaster
    \else
      \nuaa@labelEn@degreedoctor
    \fi \bigskip

    \ifdefempty{\nuaa@valueEn@applydate}{\nuaa@dateEn}{\nuaa@valueEn@applydate}
  \end{center}
}
%    \end{macrocode}
% \subsubsection{硕/博士承诺书}
%    \begin{macrocode}
\newcommand{\nuaa@make@declare@master}{
  \cleardoublepage

  \begin{center}
  \linespread{1.0}\songti\fontsize{22bp}{62.4bp}
  \vspace*{25.5bp} \vspace{-\parskip}\vspace{-\baselineskip}
  承诺书 \par
  \end{center}

  \begingroup
  \vspace*{31.4bp} \vspace{-\parskip}\vspace{-\baselineskip}
  \linespread{1.0}\songti\fontsize{16bp}{30bp}\selectfont
  本人声明所呈交的\nuaa@worktypecn 是本人在导师指导下进行的研究工作及取得的研究成果。
  除了文中特别加以标注和致谢的地方外,论文中不包含其他人已经发表或撰写过的研究成果,
  也不包含为获得\nuaa@label@nuaa 或其他教育机构的学位或证书而使用过的材料。

  本人授权\nuaa@label@nuaa 可以将学位论文的全部或部分内容编入有关数据库进行检索,
  可以采用影印、缩印或扫描等复制手段保存、汇编学位论文。

  (保密的学位论文在解密后适用本承诺书)

  \endgroup

  \vfill
  \begin{flushright}
  \linespread{1.0}\songti\fontsize{14bp}{25bp}\selectfont
  \makebox[5\ccwd][s]{作者签名:} \underline{\hspace{7em}} \par
  \makebox[5\ccwd][s]{日 \hfill 期:} \underline{\hspace{7em}} \par
  \end{flushright}
  \vfill
}
%    \end{macrocode}
% \subsubsection{硕/博士中文摘要}
%    \begin{macrocode}
\newcommand{\nuaa@make@abstract@master@cn}{
  \ifnuaa@abstractopenright
    \cleardoublepage
  \else
    \clearpage
  \fi

  \chapter*{\nuaa@label@abstract}

  \begingroup
    \nuaa@abstract
  \endgroup
  \vspace{2\baselineskip}

  \nuaa@put@kw{\textbf{\songti\nuaa@label@keywords}}{\nuaa@keywords}
}
%    \end{macrocode}
% \subsubsection{硕/博士英文摘要}
%    \begin{macrocode}
\newcommand{\nuaa@make@abstract@master@en}{
  \ifnuaa@abstractopenright
    \cleardoublepage
  \else
    \clearpage
  \fi

  \chapter*{\textbf{\nuaa@labelEn@ABSTRACT}}

  \begingroup
    \nuaa@abstractEn
  \endgroup
  \vspace{2\baselineskip}

  \nuaa@put@kw{\textbf{\nuaa@labelEn@keywords}}{\nuaa@keywordsEn}
}
%</cls>
%    \end{macrocode}
%
% \iffalse
%    \begin{macrocode}
%<*dtx-style>
\ProvidesPackage{dtx-style}
\RequirePackage[bottom,perpage,hang,]{footmisc}
\RequirePackage{hypdoc}
\RequirePackage[UTF8,scheme=chinese]{ctex}
\RequirePackage{newpxtext}
\RequirePackage{newpxmath}
\RequirePackage[
top=2.5cm, bottom=2.5cm,
left=4cm, right=2cm,
headsep=3mm]{geometry}
\RequirePackage{array,longtable,booktabs}
\RequirePackage{listings}
\RequirePackage{fancyhdr}
\RequirePackage{xcolor}
\RequirePackage{enumitem}
\RequirePackage{etoolbox}
\RequirePackage{metalogo}
\RequirePackage{graphicx}
\RequirePackage{xspace}
\RequirePackage{pifont}

\def\footnoterule{\vskip-3\p@\hrule\@width0.3\textwidth\@height0.4\p@\vskip2.6\p@}
\let\cqu@footnotesize\footnotesize
\renewcommand{\footnotesize}{\cqu@footnotesize\zihao{-5}}
\footnotemargin1.5em\relax

\let\cqu@makefnmark\@makefnmark
\def\cqu@@makefnmark{\mbox{{\normalfont\@thefnmark}}}
\pretocmd{\@makefntext}{\let\@makefnmark\cqu@@makefnmark}{}{}
\apptocmd{\@makefntext}{\let\@makefnmark\cqu@makefnmark}{}{}

\colorlet{cqu@macro}{blue!60!black}
\colorlet{cqu@env}{blue!70!black}
\colorlet{cqu@option}{purple}
\patchcmd{\PrintMacroName}{\MacroFont}{\MacroFont\bfseries\color{cqu@macro}}{}{}
\patchcmd{\PrintDescribeMacro}{\MacroFont}{\MacroFont\bfseries\color{cqu@macro}}{}{}
\patchcmd{\PrintDescribeEnv}{\MacroFont}{\MacroFont\bfseries\color{cqu@env}}{}{}
\patchcmd{\PrintEnvName}{\MacroFont}{\MacroFont\bfseries\color{cqu@env}}{}{}

\appto{\TeX}{\xspace}
\appto{\LaTeX}{\xspace}
\appto{\XeTeX}{\xspace}

\def\DescribeOption{%
    \leavevmode\@bsphack\begingroup\MakePrivateLetters%
    \Describe@Option}
\def\Describe@Option#1{\endgroup
    \marginpar{\raggedleft\PrintDescribeOption{#1}}%
    \cqu@special@index{option}{#1}\@esphack\ignorespaces}
\def\PrintDescribeOption#1{\strut \MacroFont\bfseries\sffamily\color{cqu@option} #1\ }
\def\cqu@special@index#1#2{\@bsphack
    \begingroup
    \HD@target
    \let\HDorg@encapchar\encapchar
    \edef\encapchar usage{%
        \HDorg@encapchar hdclindex{\the\c@HD@hypercount}{usage}%
    }%
    \index{#2\actualchar{\string\ttfamily\space#2}
        (#1)\encapchar usage}%
    \index{#1:\levelchar#2\actualchar
        {\string\ttfamily\space#2}\encapchar usage}%
    \endgroup
    \@esphack}

\lstdefinestyle{lstStyleBase}{%
    basicstyle=\small\ttfamily,
    aboveskip=\medskipamount,
    belowskip=\medskipamount,
    lineskip=0pt,
    boxpos=c,
    showlines=false,
    extendedchars=true,
    upquote=true,
    tabsize=2,
    showtabs=false,
    showspaces=false,
    showstringspaces=false,
    numbers=none,
    linewidth=\linewidth,
    xleftmargin=4pt,
    xrightmargin=0pt,
    resetmargins=false,
    breaklines=true,
    breakatwhitespace=false,
    breakindent=0pt,
    breakautoindent=true,
    columns=flexible,
    keepspaces=true,
    gobble=2,
    framesep=3pt,
    rulesep=1pt,
    framerule=1pt,
    backgroundcolor=\color{gray!5},
    stringstyle=\color{green!40!black!100},
    keywordstyle=\bfseries\color{blue!50!black},
    commentstyle=\slshape\color{black!60}}

\lstdefinestyle{lstStyleShell}{%
    style=lstStyleBase,
    frame=l,
    rulecolor=\color{blue},
    language=bash}

\lstdefinestyle{lstStyleLaTeX}{%
    style=lstStyleBase,
    frame=l,
    rulecolor=\color{cyan},
    language=[LaTeX]TeX}

\lstnewenvironment{latex}{\lstset{style=lstStyleLaTeX}}{}
\lstnewenvironment{shell}{\lstset{style=lstStyleShell}}{}

\setlist{nosep}

\DeclareDocumentCommand{\option}{m}{\textsf{#1}\xspace}
\DeclareDocumentCommand{\env}{m}{\texttt{#1}\xspace}
\DeclareDocumentCommand{\pkg}{s m}{%
    \texttt{#2}\xspace\IfBooleanF#1{\cqu@special@index{package}{#2}}}
\DeclareDocumentCommand{\file}{s m}{%
    \texttt{#2}\xspace\IfBooleanF#1{\cqu@special@index{file}{#2}}}
\newcommand{\myentry}[1]{%
    \marginpar{\raggedleft\color{purple}\bfseries\strut #1}}
\newcommand{\note}[1]{{%
        \color{magenta}{\noindent\bfseries 说明:}\emph{#1}}}

\def\cquthesis{\textsc{Cqu}\-\textsc{Thesis}}
%</dtx-style>
%    \end{macrocode}
% \fi
%
% \Finale
\endinput
