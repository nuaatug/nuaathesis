\iffalse
  % 本块代码被上方的 iffalse 注释掉,如需使用,请改为 iftrue
  % 使用 Noto 字体替换中文宋体、黑体
  \setCJKfamilyfont{\CJKrmdefault}[BoldFont=Noto Serif CJK SC Bold]{Noto Serif CJK SC}
  \renewcommand\songti{\CJKfamily{\CJKrmdefault}}
  \setCJKfamilyfont{\CJKsfdefault}[BoldFont=Noto Sans CJK SC Bold]{Noto Sans CJK SC Medium}
  \renewcommand\heiti{\CJKfamily{\CJKsfdefault}}
\fi

\iffalse
  % 本块代码被上方的 iffalse 注释掉,如需使用,请改为 iftrue
  % 在 XeLaTeX + ctexbook 环境下使用 Noto 日文字体
  \setCJKfamilyfont{mc}[BoldFont=Noto Serif CJK JP Bold]{Noto Serif CJK JP}
  \newcommand\mcfamily{\CJKfamily{mc}}
  \setCJKfamilyfont{gt}[BoldFont=Noto Sans CJK JP Bold]{Noto Sans CJK JP}
  \newcommand\gtfamily{\CJKfamily{gt}}
\fi


% 设置基本文档信息,\linebreak 前面不要有空格,否则在无需换行的场合,中文之间的空格无法消除
\nuaaset{
  thesisid = {1028704 18-S000},   % 论文编号
  title = {\nuaathesis{} 快速上手\linebreak 示例文档},
  author = {nuaatug},
  college = {\TeX{} 学院},
  advisers = {Donald Knuth\quad 大师, tex.se 大牛们},
  % applydate = {二〇一八年六月}  % 默认当前日期
  %
  % 本科
  major = {\LaTeX{} 科学与技术},
  studentid = {131810299},
  classid = {应用技术},           % 班级的名称
  industrialadvisers = {Jack Ma}, % 企业导师,若无请删除或注释本行
  % 硕/博士
  majorsubject = {\LaTeX},
  researchfield = {\LaTeX 排版},
  libraryclassid = {TP371},       % 中图分类号
  subjectclassid = {080605},      % 学科分类号
}
\nuaasetEn{
  title = {\nuaathesis{} Quick Start\linebreak and Document Snippets},
  author = {nuaatug},
  college = {College of \TeX},
  majorsubject = {\LaTeX{} Typesetting},
  advisers = {Prof.~Donald Knuth, tex.se users},
  degreefull = {Master of Art and Engineering},
  % applydate = {June, 8012}
}

% 摘要
\begin{abstract}
本文介绍如何使用\nuaathesis{} 文档类撰写南京航空航天大学学位论文。

首先介绍如何获取并编译本文档,然后展示论文部件的实例,最后列举部分常用宏包的使用方法。
\end{abstract}
\keywords{学位论文, 模板, \nuaathesis}

\begin{abstractEn}
This document introduces \nuaathesis, the \LaTeX{} document class for NUAA Thesis.

First, we show how to get the source code and compile this document.
Then we provide snippets for figures, tables, equations, etc.
Finally we enforce some usage patterns.
\end{abstractEn}
\keywordsEn{NUAA thesis, document class, space is accepted here}


% 请按自己的论文排版需求,随意修改以下全局设置

\usepackage{subfig}
\usepackage{rotating}
\usepackage[usenames,dvipsnames]{xcolor}
\usepackage{tikz}
\usepackage{pgfplots}
\pgfplotsset{compat=1.16}
\pgfplotsset{
  table/search path={./fig/},
}
\usepackage{ifthen}
\usepackage{longtable}
\usepackage{siunitx}
\usepackage{listings}
\usepackage{multirow}
\usepackage{pifont}

\lstdefinestyle{lstStyleBase}{%
  basicstyle=\small\ttfamily,
  aboveskip=\medskipamount,
  belowskip=\medskipamount,
  lineskip=0pt,
  boxpos=c,
  showlines=false,
  extendedchars=true,
  upquote=true,
  tabsize=2,
  showtabs=false,
  showspaces=false,
  showstringspaces=false,
  numbers=left,
  numberstyle=\footnotesize,
  linewidth=\linewidth,
  xleftmargin=\parindent,
  xrightmargin=0pt,
  resetmargins=false,
  breaklines=true,
  breakatwhitespace=false,
  breakindent=0pt,
  breakautoindent=true,
  columns=flexible,
  keepspaces=true,
  framesep=3pt,
  rulesep=2pt,
  framerule=1pt,
  backgroundcolor=\color{gray!5},
  stringstyle=\color{green!40!black!100},
  keywordstyle=\bfseries\color{blue!50!black},
  commentstyle=\slshape\color{black!60}}

%\usetikzlibrary{external}
%\tikzexternalize % activate!

\newcommand\cs[1]{\texttt{\textbackslash#1}}
\newcommand\pkg[1]{\texttt{#1}\textsuperscript{PKG}}
\newcommand\env[1]{\texttt{#1}}

\theoremstyle{nuaaplain}
\nuaatheoremchapu{definition}{定义}
\nuaatheoremchapu{assumption}{假设}
\nuaatheoremchap{exercise}{练习}
\nuaatheoremchap{nonsense}{胡诌}
\nuaatheoremg[句]{lines}{句子}
