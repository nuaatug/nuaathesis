% 本文件是示例论文的一部分
% 论文的主文件位于上级目录的 `bachelor.tex` 或 `master.tex`

\chapter{快速上手}

\section{欢迎}

欢迎使用 \nuaathesis,本文档将介绍如何利用 \nuaathesis 模板进行学位论文写作,
我们假设读者有 \LaTeX 英文写作经验,并会使用搜索引擎解决常见问题。

本模板的源代码托管在 \url{https://github.com/nuaatug/nuaathesis},
欢迎来提 issue/PR。

\section{\LaTeX 环境准备}

由于\nuaathesis 使用了一些不太常用的宏包,\TeX~Live 用户需要手动安装后才能使用本模板。

如果使用系统(Debian, ArchLinux 等)的 \TeX~Live 软件包的话,请用系统的软件包管理器,安装以下 collection:
\verb|langchinese, latexextra, science, pictures, fontsextra|.

如果是从 \TeX~Live 官网下载光盘镜像、或是在线安装的话,
\begin{enumerate}
  \item 首先运行 \verb|tlmgr --help| 来确认 \TeX~Live 环境是否就绪,它应该会输出很长的帮助说明;
  \item 在本模板的 \verb|.ci| 目录(可能是隐藏目录),运行 \verb|install.bat| (Windows) 或 \verb|install.sh|,
  它会利用 \verb|tlmgr| 下载安装所需的宏包;
  \item Windows 的脚本依赖于 PowerShell,如果没有安装该系统功能的话,脚本无法使用;
  如需手工安装的话,可以参考 \verb|.ci/texlive.pkgs| 中列出的宏包清单。
\end{enumerate}

如果使用 Mik\TeX 的话,理论上它会自动下载安装缺失的宏包,good luck :)

\section{编译模板和文档}

进入模板的根目录,运行 \verb|build.bat|(Windows) 或 \verb|build.sh|(其他系统),
它会生成模板 \verb|nuaathesis.cls| 以及对应的文档 \verb|nuaathesis.pdf|。

\section{使用模板}

论文写作时,请确认工作目录下有以下文件:
\begin{itemize}
  \item \verb|nuaathesis.cls| 文档模板;
  \item \verb|nuaathesis.bst| 参考文献格式(如果使用 biber 来生成参考文献的话);
  \item \verb|logo/| 文件夹,内含一些图标;
\end{itemize}

如果工作目录下没有这些文件的话,请从本模板根目录复制一份。

\section{开始写作}

最方便的开始方法,莫过于在修改现有的文章,特别是这篇快速手册。需要修改的文件有:
\begin{itemize}
  \item \verb|master.tex| 主文件,定义了文档包含的内容;
  \item \verb|global.tex| 里面定义文档的信息,导入一些宏包,并设置全局使用的宏;
  \item \verb|content/| 文件夹,按章节拆分的文档内容;
  \item \verb|ref/| 文件夹,内含参考文献数据库;
\end{itemize}

修改完成后,使用 \verb|latexmk -xelatex master| 进行编译。

当然,主文件也可能叫作 \verb|bachelor.tex| 或其他名字,请对照着修改编译命令。
