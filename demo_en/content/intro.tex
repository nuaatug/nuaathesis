\chapter{INTRODUCTION}

\section{Theorem Environment}
Here is the \autoref{def:bonus}.
\begin{definition}\label{def:bonus}
  Bonus points are extra gains.
\end{definition}

\section{Using Figure}
Take a look at the figure below, and the table of figures for the \autoref{fig:logo}.

\begin{figure}[H]
  \includegraphics[width=3cm]{nuaa-logo}
  \caption[Demo logo]{This is a logo\label{fig:logo}}
\end{figure}

\section{Using Table}
Never underestimate the power of crowd (in bus/metro) as shown in \autoref{tab:city}.
\begin{table}[htb]
  \caption[City population]{City with large population (source: Wikipedia)\label{tab:city}}
  \begin{tabular}{lr}
    \toprule
    City & Population \\
    \midrule
    Mexico City & 20,116,842\\
    Shanghai & 19,210,000\\
    Peking & 15,796,450\\
    Istanbul & 14,160,467\\
    \bottomrule
  \end{tabular}
\end{table}

\section{Using Proof}
\begin{proof}
It is impossible to separate a cube into two cubes, or a fourth power into two fourth powers,
or in general, any power higher than the second, into two like powers.

I have discovered a truly marvelous proof of this, which this margin is too narrow to contain.
\end{proof}

\section{Using Reference}
Cite one paper\cite{r1}, or multiple\cite{r2,r3,r4}.

Here is inline cited paper\inlinecite{r6}, and another paper\inlinecite{r7,r8,r9}.

\section{Organization of the Thesis}
The organization of this thesis is as follows.
